% \item Проставить ссылки на источники
% \item Оглавление
% \item Отчёт

% \item записать постановку задачи
% \item описать способы запуска и взаимодействие с тестами

% \item Сравнить циклическую редукцию с остальными методами. Какая сложность?
% \item Дописать выкладки
% \item Радиальная гистограмма?
% \item Метод зонтика для FES Метод взвешенных гистограмм

%   Входные данные:
%   размеры шагов по времени $\tau$ и пространству $h$, коэффициенты $D$, $q = \sigma(s) \cdot E^2$, начальное условия $U(r, 0) = U_0(r)$, граничные условия $U|_\Omega = g(r,t)$


\documentclass[a4paper,12pt]{article}
\usepackage[T2A]{fontenc}
\usepackage[utf8]{inputenc}
\usepackage[russianb,english]{babel}
\usepackage[pdftex,unicode]{hyperref}
\usepackage{amssymb,amsfonts,amsmath,mathtext,cite,enumerate,float,indentfirst}

\usepackage[fleqn]{mathtools}
\usepackage{geometry} % Меняем поля страницы
\usepackage{cmap} % Русский поиск в pdf
\usepackage{ccaption}
\captiondelim{. } % после точки стоит пробел!
\geometry{left=2cm} % левое поле
\geometry{right=2cm}% правое поле
\geometry{top=2cm}% верхнее поле
\geometry{bottom=2cm}% нижнее поле
\usepackage{graphicx}
\graphicspath{{images/}}

\sloppy

\begin{document}
\renewcommand{\contentsname}{Содержание}
\renewcommand{\figurename}{Рис.}
\renewcommand{\bibname}{Список литературы}
\renewcommand{\refname}{Список литературы}
\renewcommand{\tablename}{Таблица}

\section{Постановка задачи для уравнения теплопроводности}
\label{sec:problem}

Рассмотрим задачу для уравнения теплопроводности.

\begin{align}
  \frac{\partial U(t,\vec r)}{\partial t} &= \nabla^2 U(t,\vec r) + H(t, \vec r),
    \qquad \vec r \in \Omega,
    \quad \partial \Omega = \partial \Omega_1 \cup \partial \Omega_2 \\
  U(0, \vec r) &=  \vec \psi (\vec r), \\
  \left. U(t,\vec r) \right|_{\vec r \in \partial \Omega_1} &=  \vec \varphi(t), \\
  \left. \frac{ \partial U(t,\vec r) }{\partial \vec r} \right|_{\vec r \in \partial \Omega_2}& +
    p(t)\left. U(t,\vec r) \right|_{\vec r \in \partial \Omega_2}  =  \vec q(t).
\end{align}

Смысл последней записи заключается в том, что на части границ области может быть задано граничаное условие первого рода, в том время, как на оставшихся границах заданы граничные условия третьего (или второго как частный случай) рода.

\section{Схема Кранка -- Николсон}
\label{sec:heat1d}

В одномерном случае задача приобретает вид:
\begin{align}
  \label{eq:1d-problem-eq}
  \frac{\partial U(t,x)}{\partial t} &= \frac{\partial^2}{\partial x^2} U(t,x) + H(t, x),
    \quad x \in \Omega = [x_0,x_1], \ \partial \Omega = \partial \Omega_1 \cup \partial \Omega_2 = \{x_0, x_1\}\\
  \label{eq:1d-problem-ic}
  U(0, x) &= \psi(x), \\
  \label{eq:1d-problem-bc1}
  \left. U(t,x) \right|_{x \in \partial \Omega_1} &=  \varphi(t), \\
  \label{eq:1d-problem-bc3}
  \left. \frac{ \partial U(t,x) }{\partial x} \right|_{x \in \partial \Omega_2}& +
    p(t)\left. U(t,x) \right|_{x \in \partial \Omega_2}  = q(t).
\end{align}

Схема Кранка--Никольсон для уравнения теплопроводности имеет вид:
\begin{equation}
  \frac{U^{n+1} - U^n}{\tau} = \frac{1}{2} L_1 U^{n+1}  + \frac{1}{2} L_1 U^n +
    \frac{1}{2} \left( H^{n+1} + H^{n} \right).
\end{equation}

Проведём разделение известных и неизвестных.
\begin{equation*}
  \left( I - \frac{\tau}{2} L_1 \right) U^{n+1} =
    \left( I + \frac{\tau}{2} L_1 \right) U^n +
    \frac{1}{2} \left( H^{n+1} + H^{n} \right).
\end{equation*}

Оператор имеет вид

\begin{equation*}
  L_1 U = \frac{D^+_x \frac{U_{i+1} - U_{i}}{h} - D^-_x\frac{U_{i} - U_{i-1}}{h}}{h} = \frac{D^+_x}{h^2}U_{i+1} - \left( \frac{D^+_x}{h^2} + \frac{D^-_x}{h^2} \right)U_{i} +  \frac{D^-_x}{h^2}U_{i-1}.
\end{equation*}
Левая часть имеет вид
\begin{multline*}
  \left( I - \frac{\tau}{2} L_1 \right) U^{n+1} = U^{n+1}_i - \frac{\tau}{2} \left[ \frac{D^+_x}{h^2}U_{i+1} - \left( \frac{D^+_x}{h^2} + \frac{D^-_x}{h^2} \right) U^{n+1}_{i} +  \frac{D^-_x}{h^2}U^{n+1}_{i-1} \right] = \\
  - \frac{\tau D^+_x}{2h^2}U^{n+1}_{i+1} + \left(1 + \frac{\tau D^+_x}{2 h^2} + \frac{\tau D^-_x}{2 h^2} \right) U^{n+1}_{i} - \frac{\tau D^-_x}{2h^2}U^{n+1}_{i-1}.
\end{multline*}
Правая часть имеет вид
\begin{multline*}
  \left( I + \frac{\tau}{2} L_1 \right) U^{n} =
    U^{n}_i + \frac{\tau}{2} \left[ \frac{D^+_x}{h^2}U^{n}_{i+1} -
    \left( \frac{D^+_x}{h^2} + \frac{D^-_x}{h^2} \right) U^{n}_{i} +
    \frac{D^-_x}{h^2}U^{n}_{i-1} \right] = \\
  \frac{\tau D^+_x}{2h^2}U^{n}_{i+1} + \left(1 - \left[\frac{\tau D^+_x}{2 h^2} +
    \frac{\tau D^-_x}{2 h^2} \right] \right) U^{n}_{i} +
    \frac{\tau D^-_x}{2h^2}U^{n}_{i-1} + \frac{1}{2} \left( H^{n+1}_i + H^{n}_i \right).
\end{multline*}
Окончательно имеем
\begin{multline*}
  - \frac{\tau D^+_x}{2h^2}U^{n+1}_{i+1} + \left(1 + \frac{\tau D^+_x}{2 h^2} + \frac{\tau D^-_x}{2 h^2} \right) U^{n+1}_{i} - \frac{\tau D^-_x}{2h^2}U^{n+1}_{i-1}= \\
  = \frac{\tau D^+_x}{2h^2}U^{n}_{i+1} + \left(1 - \left[ \frac{\tau D^+_x}{2 h^2} + \frac{\tau D^-_x}{2 h^2} \right] \right) U^{n}_{i} + \frac{\tau D^-_x}{2h^2}U^{n}_{i-1} + \frac{1}{2} \left( H^{n+1}_i + H^{n}_i \right), \quad i = 1, \ldots, N_x.
\end{multline*}

Обозначим  $a_x = \frac{\tau D^+_x}{2h^2}, c_x = \frac{\tau D^-_x}{2h^2}, b_x = a_x + c_x$. Тогда
\begin{equation}
  \label{eq:1d-scheme}
  - a_x U^{n+1}_{i+1} + \left(1 + b_x \right) U^{n+1}_{i} - c_x U^{n+1}_{i-1} = a_x U^{n}_{i+1} + \left(1 - b_x \right) U^{n}_{i} + c_x U^{n}_{i-1}.
\end{equation}
где $i = \overline{1, N_x-1}, a_x = \frac{\tau D^+_x}{2h^2}, c_x = \frac{\tau D^-_x}{2h^2}, b_x = a_x + c_x$.

Начальные условия
\begin{equation}
  \label{eq:ic}
  U^0_i = \psi_i, \quad i = \overline {1, N_x-1}.
\end{equation}

Для граничных условий первого рода никакой модификации не требуется:
\begin{equation}
  \label{eq:1d-bc1}
  U^n_0 = \varphi^n, \qquad \mbox{или} \qquad U^n_{N_x} = \varphi^n, \qquad n = \overline{0,T}
\end{equation}

Рассмотрим примеры граничных условий третьего рода
\begin{equation*}
  \left. \frac{ \partial U(t,x) }{\partial x} \right|_{x = x_0} +
  p(t) U(t, x_0) = q(t), \qquad
  \left. \frac{ \partial U(t, x) }{\partial x} \right|_{x = x_1} +
  p(t) U(t, x_1) = q(t).
\end{equation*}

По формуле центральной разности для граничного условия на левом конце имеем
\begin{equation*}
  \frac{U^{n}_{1} -  U^{n}_{-1}}{2 h_x} + p^n U^n_{0}  = q^{n}, \qquad
  U^{n+1}_{-1}  = U^{n}_{1} - 2 h_x \left( p^n U^n_{0} + q^{n} \right)
\end{equation*}
Подставим в уравнение
\begin{multline*}
  - a_x U^{n+1}_{1} + \left( 1 + b_x  \right) U^{n+1}_{0} - c_x \left( U^{n+1}_{1} - 2 h_x (p^{n+1} U^{n+1}_{0} + q^{n+1}) \right) = \\
  = a_x U^{n}_{1} + \left(1 - b_x \right) U^{n}_{0} + c_x \left(  U^{n}_{1} - 2 h_x (p^n U^{n}_{0} + q^{n}) \right).
\end{multline*}
\begin{multline*}
  - (a_x + c_x) U^{n+1}_{1} + \left( 1 + b_x + 2 c_x h_x p^{n+1} \right) U^{n+1}_{0} = \\
  = (a_x + c_x) U^{n}_{1} + \left(1 - b_x - 2 c_x h_x p^n \right) U^{n}_{0} - 2 c_x h_x (q^{n+1} + q^{n}).
\end{multline*}
\begin{equation}
  - b_x U^{n+1}_{1} + \left( 1 + b_x  + 2 c_x h_x p^{n+1} \right) U^{n+1}_{0} =
  b_x U^{n}_{1} + \left(1 - b_x - 2 c_x h_x p^n \right) U^{n}_{0} - 2 c_x h_x (q^{n+1} + q^{n}).
\end{equation}

Обозначим $\alpha = 1 + b_x  + 2 c_x h_x p^{n+1}, \quad \beta = - b_x$
\begin{equation}
  \label{eq:1d-bc3-x0}
  \beta U^{n+1}_{1} + \alpha U^{n+1}_{0} =
  b_x U^{n}_{1} + \left(1 - b_x - 2 c_x h_x p^n \right) U^{n}_{0} - 2 c_x h_x (q^{n+1} + q^{n}).
\end{equation}

Аналогично для правого конца
\begin{equation*}
  \frac{U^{n}_{N_x+1} -  U^{n}_{N_x-1}}{2 h_x} + p^n U^n_{N_x}  = q^{n}, \qquad
  U^{n}_{N_x+1}  = U^{n}_{N_x-1} - 2 h_x \left( p^n U^n_{N_x} - q^{n} \right)
\end{equation*}

\begin{multline*}
  - a_x \left( U^{n+1}_{N_x-1} - 2 h_x \left( p^{n+1} U^{n+1}_{N_x} - q^{n+1} \right) \right) + \left( 1 + b_x \right) U^{n+1}_{N_x} - c_x U^{n+1}_{N_x-1} = \\
  = a_x \left(U^{n}_{N_x-1} - 2 h_x \left( p^n U^n_{N_x} - q^{n} \right) \right) + \left(1 - b_x  \right) U^{n}_{N_x} + c_x U^{n}_{N_x-1}.
\end{multline*}
\begin{equation*}
  \left( 1 + b_x + 2 h_x a_x p^{n+1} \right) U^{n+1}_{N_x}  - (a_x + c_x) U^{n+1}_{N_x-1}  = \left(1 - b_x - 2 h_x a_x p^{n} \right) U^{n}_{N_x} + (a_x + c_x) U^{n}_{N_x-1} + 2 a_x h \left( q^{n+1} + q^{n} \right).
\end{equation*}

\begin{equation}
  \left( 1 + b_x + 2 h_x a_x p^{n+1} \right) U^{n+1}_{N_x}  - b_x U^{n+1}_{N_x-1}  = \left(1 - b_x - 2 h_x a_x p^{n} \right) U^{n}_{N_x} + b_x U^{n}_{N_x-1} + 2 a_x h_x \left( q^{n+1} + q^{n} \right).
\end{equation}

Обозначим $\gamma = - b_x, \quad \delta = 1 + b_x + 2 h_x a_x p^{n+1}$
\begin{equation}
  \label{eq:1d-bc3-x1}
  \delta U^{n+1}_{N_x} + \gamma U^{n+1}_{N_x-1}  = \left(1 - b_x - 2 h_x a_x p^{n+1} \right) U^{n}_{N_x} + b_x U^{n}_{N_x-1} + 2 a_x h_x \left( q^{n+1} + q^{n} \right).
\end{equation}


\hrulefill

Объединяя формулы (\ref{eq:1d-scheme}) -- (\ref{eq:1d-bc3-x1}) заключаем, что задача (\ref{eq:1d-problem-eq}) -- (\ref{eq:1d-problem-bc3}) сводится к следующему виду:
\begin{equation*}
  - a_x U^{n+1}_{i+1} + \left(1 + b_x \right) U^{n+1}_{i} - c_x U^{n+1}_{i-1} = a_x U^{n}_{i+1} + \left(1 - b_x \right) U^{n}_{i} + c_x U^{n}_{i-1} + \frac{1}{2} \left( H^{n+1}_i + H^{n}_i \right).
\end{equation*}
где $i = \overline {1, N_x-1}, \ n = \overline{0,T}; \ a_x = \frac{\tau D^+_x}{2h^2}, c_x = \frac{\tau D^-_x}{2h^2}, b_x = a_x + c_x$.

Начальные условия
\begin{equation*}
  U^0_i = \psi_i, \quad i = \overline {1, N_x-1}.
\end{equation*}
Примеры граничных условий первого рода:
\begin{equation*}
  U^n_0 = \varphi_1^n, \qquad U^n_{N_x} = \varphi_2^n, \qquad n = \overline{0,T}.
\end{equation*}
Примеры граничных условий третьего рода:
\begin{align*}
  \beta U^{n+1}_{1} + \alpha U^{n+1}_{0} &=
  b_x U^{n}_{1} + \left(1 - b_x - 2 c_x h_x p^n \right) U^{n}_{0} - 2 c_x h_x (q^{n+1} + q^{n}). \\
  \delta U^{n+1}_{N_x} + \gamma U^{n+1}_{N_x-1} &=
  \left(1 - b_x - 2 h_x a_x p^{n+1} \right) U^{n}_{N_x} + b_x U^{n}_{N_x-1} + 2 a_x h_x \left( q^{n+1} + q^{n} \right).
\end{align*}
где $n = \overline{0,T}$,
$\alpha = 1 + b_x  + 2 c_x h_x p^n, \quad \beta = - b_x$, $\gamma = - b_x, \quad \delta =  1 + b_x + 2 h_x a_x $

\hrulefill

\section{Схема Дугласа -- Ганна для двумерной прямоугольной области}
\label{sec:heat2d}

В двумерном случае задача приобретает вид:
\begin{align}
  \label{eq:2d-problem-eq}
  \frac{\partial U(t,x,y)}{\partial t} &= \frac{\partial^2}{\partial x^2} U(t,x,y) + \frac{\partial^2}{\partial y^2} U(t,x,y) + H(t,x,y),
                                         \quad x \in \Omega = [x_0,x_1] \times [y_0, y_1]\\
  \label{eq:2d-problem-ic}
  U(0, x, y) &= \psi(x, y),\\
  \label{eq:2d-problem-bc1}
  \left. U(t,x, y) \right|_{(x,y) \in \partial \Omega_1} &=  \varphi(t, x, y),   \ \partial \Omega = \partial \Omega_1 \cup \partial \Omega_2 \\
  \label{eq:2d-problem-bc3}
  \left. \frac{ \partial U(t,x,y) }{\partial \vec n} \right|_{(x,y) \in \partial \Omega_2}& +
                                                                                            p(t)\left. U(t,x,y) \right|_{(x,y) \in \partial \Omega_2}  = q(t, x, y).
\end{align}


Схема Дугласа -- Ганна для двумерной области
\begin{equation*}
  \begin{cases}
    \frac{U^* - U^n}{\tau} = \frac{1}{2} L_1 U^* + \frac{1}{2} L_1 U^n + L_2 U^n + H^n, \\
    \frac{U^{n+1} - U^n}{\tau} = \frac{1}{2} L_1 U^* + \frac{1}{2} L_1 U^n + \frac{1}{2} L_2 U^{n+1} + \frac{1}{2} L_2 U^n + H^{n+1}.
  \end{cases}
\end{equation*}

Проведём разделение известных и неизвестных переменных для каждого из уравнений.
\begin{equation*}
  \begin{cases}
    \frac{U^*}{\tau} - \frac{1}{2} L_1 U^* = \frac{U^n}{\tau} + \frac{1}{2} L_1 U^n + L_2 U^n, \\
    \frac{U^{n+1}}{\tau} - \frac{1}{2} L_2 U^{n+1} = \frac{U^n}{\tau} + \frac{1}{2} L_1 U^* + \frac{1}{2} L_1 U^n + \frac{1}{2} L_2 U^n.
  \end{cases}
\end{equation*}

Сгруппируем операторы
\begin{equation*}
  \begin{cases}
    \left( \frac{1}{\tau} I - \frac{1}{2} L_1 \right) U^* = \left( \frac{1}{\tau} I + \frac{1}{2} L_1 + L_2 \right) U^n, \\
    \left( \frac{1}{\tau} I - \frac{1}{2} L_2 \right) U^{n+1} = \left( \frac{1}{\tau} I  + \frac{1}{2} L_1 + \frac{1}{2} L_2 \right) U^n + \frac{1}{2} L_1 U^*
  \end{cases}
\end{equation*}

Вычтем первое уравнение из второго:
\begin{equation*}
  \begin{cases}
    \left( \frac{1}{\tau} I - \frac{1}{2} L_1 \right) U^* = \left( \frac{I}{\tau} + \frac{1}{2} L_1 + L_2 \right) U^n, \\
    \left( \frac{1}{\tau} I - \frac{1}{2} L_2 \right) U^{n+1} - \left( \frac{1}{\tau} I - \frac{1}{2} L_1 \right) U^* = - \frac{1}{2} L_2 U^n + \frac{1}{2} L_1 U^*
  \end{cases}
\end{equation*}

Отсюда
\begin{equation*}
  \begin{cases}
    \left( \frac{1}{\tau} I - \frac{1}{2} L_1 \right) U^* = \left( \frac{I}{\tau} + \frac{1}{2} L_1 + L_2 \right) U^n, \\
    \left( \frac{1}{\tau} I - \frac{1}{2} L_2 \right) U^{n+1} = \frac{1}{\tau} U^* - \frac{1}{2} L_2 U^n
  \end{cases}
\end{equation*}

Домножим на $\tau$
\begin{equation*}
  \begin{cases}
    \left( I - \frac{\tau}{2} L_1 \right) U^* = \left( I + \frac{\tau}{2} L_1 + \tau L_2 \right) U^n, \\
    \left( I - \frac{\tau}{2} L_2 \right) U^{n+1} = U^* - \frac{\tau}{2} L_2 U^n, \\
  \end{cases}
\end{equation*}

Операторы имеют вид:
\begin{multline*}
  L_1 U = \frac{D^+_x \frac{U_{i+1,j} - U_{i,j}}{h_x} - D^-_x\frac{U_{i,j} - U_{i-1,j}}{h_x}}{h_x}
  = \frac{D^+_x}{h_x^2} U_{i+1,j} - \frac{D^+_x}{h_x^2}U_{i,j} - \frac{D^-_x}{h_x^2} U_{i,j} + \frac{D^-_x}{h_x^2}U_{i-1,j} \quad \forall i,j;
\end{multline*}
\begin{equation*}
  L_2 U = \frac{D^+_y}{h_y^2} U_{i,j+1} - \frac{D^+_y}{h_y^2}U_{i,j} - \frac{D^-_y}{h_y^2} U_{i,j} + \frac{D^-_y}{h_y^2}U_{i,j-1} \quad \forall i,j;
\end{equation*}

Преобразуем левые части уравнений
\begin{multline*}
  \left( I - \frac{\tau}{2} L_1 \right) U^* = U^*_{i,j} - \frac{\tau}{2} \left[\frac{D^+_x}{h_x^2} U^*_{i+1,j} - \frac{D^+_x}{h_x^2}U^*_{i,j} - \frac{D^-_x}{h_x^2} U^*_{i,j} + \frac{D^-_x}{h_x^2}U^*_{i-1,j} \right] = \\
  - \frac{\tau D^+_x }{2 h_x^2} U^*_{i+1,j} + \left(1 + \frac{\tau D^+_x}{2h_x^2} + \frac{\tau D^-_x}{2 h_x^2} \right) U^*_{i,j} - \frac{\tau D^-_x}{2h_x^2} U^*_{i-1,j}
\end{multline*}

\begin{multline*}
  \left( I - \frac{\tau}{2} L_2 \right) U^{n+1} = U^{n+1}_{i,j} - \frac{\tau}{2} \left[ \frac{D^+_y}{h_y^2} U^{n+1}_{i,j+1} - \frac{D^+_y}{h_y^2}U^{n+1}_{i,j} - \frac{D^-_y}{h_y^2} U^{n+1}_{i,j} + \frac{D^-_y}{h_y^2}U^{n+1}_{i,j-1}  \right] = \\
  - \frac{\tau D^+_y }{2 h_y^2} U^{n+1}_{i,j+1} + \left(1 + \frac{\tau D^+_y}{2h_y^2} + \frac{\tau D^-_y}{2 h_y^2} \right) U^{n+1}_{i,j} - \frac{\tau D^-_y}{2h_y^2} U^{n+1}_{i,j-1}
\end{multline*}

Обозначим  $a_x = \frac{\tau D^+_x}{2h_x^2}, c_x = \frac{\tau D^-_x}{2h_x^2}$ и  $a_y = \frac{\tau D^+_y}{2h_y^2}, c_y = \frac{\tau D^-_y}{2h_y^2}$ соответственно. Тогда
\begin{equation*}
  \left( I - \frac{\tau}{2} L_1 \right) U^* = - a_x U^*_{i+1,j} + \left(1 + a_x + c_x \right) U^*_{i,j} - c_x U^*_{i-1,j}
\end{equation*}
\begin{equation*}
  \left( I - \frac{\tau}{2} L_2 \right) U^{n+1} = - a_y U^{n+1}_{i,j+1} + \left(1 + a_y + c_y \right) U^{n+1}_{i,j} - c_y U^{n+1}_{i,j-1}
\end{equation*}

Преобразуем правые части:
\begin{multline*}
  \left( I + \frac{\tau}{2} L_1 + \tau L_2 \right) U^n = U^n_{i,j} + \frac{\tau}{2} \left[  \frac{D^+_x}{h_x^2} U^n_{i+1,j} - \frac{D^+_x}{h_x^2}U^n_{i,j} - \frac{D^-_x}{h_x^2} U^n_{i,j} + \frac{D^-_x}{h_x^2}U^n_{i-1,j} \right] + \\  + \tau \left[ \frac{D^+_y}{h_y^2} U^n_{i,j+1} - \frac{D^+_y}{h_y^2}U^n_{i,j} - \frac{D^-_y}{h_y^2} U^n_{i,j} + \frac{D^-_y}{h_y^2}U^n_{i,j-1} \right] = \frac{\tau D^+_x}{2 h_x^2} U^n_{i+1,j} + \frac{\tau D^-_x}{2 h_x^2} U^n_{i-1,j} + \\ + \frac{\tau D^+_y}{h_y^2} U^n_{i,j+1} + \frac{\tau D^-_y}{h_y^2} U^n_{i,j-1} + \left[ 1 -  \frac{\tau D^+_x}{2h_x^2} - \frac{\tau D^-_x}{2h_x^2} - \frac{ \tau D^+_y}{h_y^2} - \frac{\tau D^-_y}{h_y^2} \right] U^n_{i,j}
\end{multline*}
\begin{multline*}
  U^* - \frac{\tau}{2} L_2 U^n = U_{i,j}^* - \frac{\tau}{2} \left[\frac{D^+_y}{h_y^2} U^{n}_{i,j+1} - \frac{D^+_y}{h_y^2}U^{n}_{i,j} - \frac{D^-_y}{h_y^2} U^{n}_{i,j} + \frac{D^-_y}{h_y^2}U^{n}_{i,j-1} \right] = \\
  = U_{i,j}^* - \frac{\tau D^+_y}{2 h_y^2} U_{i,j+1} - \frac{\tau D^-_y}{2 h_y^2} U_{i,j-1} + \left[ \frac{\tau D^+_y}{2 h_y^2} + \frac{\tau D^-_y}{2 h_y^2} \right] U_{i,j}.
\end{multline*}

Или, используя ранее введённые обозначения
\begin{multline*}
  \left( I + \frac{\tau}{2} L_1 + \tau L_2 \right) U^n =
  a_x U^n_{i+1,j} + c_x U^n_{i-1,j} + 2  a_y U^n_{i,j+1} + 2 c_y U^n_{i,j-1} +
  (1 -  a_x - c_x - 2 a_y - 2 c_y ) U^n_{i,j}
\end{multline*}
\begin{multline*}
  U^* - \frac{\tau}{2} L_2 U^n = U_{i,j}^* - a_y U^n_{i,j+1} - c_y U^n_{i,j-1} +
  \left( a_y + c_y \right) U^n_{i,j} = \\
  = U_{i,j}^* - a_y U^n_{i,j+1} - c_y U^n_{i,j-1} + (a_y + c_y) U^n_{i,j}.
\end{multline*}


Обозначая $b_x = a_x + c_x$ и $b_y = a_y + c_y$
\begin{align}
  \label{eq:2d-scheme}
  - a_x U^*_{i+1,j} + \left(1 + b_x \right) U^*_{i,j} - c_x  U^*_{i-1,j} =
  a_x U^n_{i+1,j} + c_x U^n_{i-1,j} + 2  a_y U^n_{i,j+1} + 2 c_y U^n_{i,j-1} +
  (1 -  b_x - 2 b_y) U^n_{i,j},\\
  - a_y U^{n+1}_{i,j+1} + \left(1 + b_y \right)  U^{n+1}_{i,j} - c_y U^{n+1}_{i,j-1} =
  U_{i,j}^* - a_y U^n_{i,j+1} - c_y U^n_{i,j-1} + b_y U^n_{i,j}.
\end{align}

Начальные условия
\begin{equation}
  \label{eq:2d-ic}
  U^0_{ij} = \psi_{ij}, \quad i = \overline {1, N_x-1}, \ j = \overline {1, N_y-1}.
\end{equation}

Примеры граничных условий первого рода:
\begin{equation*}
  \label{eq:2d-bc1}
  U^n_{0j} = \varphi_j^n, \qquad U^n_{N_x j} = \varphi_j^n, \qquad  U^n_{i0} = \varphi_i^n, \qquad U^n_{i N_y} = \varphi_i^n,
\end{equation*}
где $j = \overline {1, N_y-1}, \ n = \overline{0,T}$.
Рассмотрим примеры граничных условий третьего рода:
\begin{align*}
  &\left. \frac{ \partial U(t,x,y) }{\partial x} \right|_{x = x_0} +
    p(t) U(t, x_0, y) = q(t, y),
    \left. \frac{ \partial U(t,x,y) }{\partial x} \right|_{x = x_1} +
    p(t) U(t, x_1, y) = q(t, y), \quad y \in [y_0,y_1] \\
  &\left. \frac{ \partial U(t,x,y) }{\partial y} \right|_{y = y_0} +
    p(t) U(t, x, y_0)= q(t, x),
    \left. \frac{ \partial U(t,x,y) }{\partial y} \right|_{y = y_1} +
    p(t) U(t, x, y_1)= q(t, x), \quad x \in [x_0,x_1]
\end{align*}

По формуле центральной разности для условия на границе $x = x_0$
\begin{equation*}
  \frac{U^{n}_{1,j} -  U^{n}_{-1,j}}{2 h_x} + p_j^n U^n_{0,j}  = q_j^{n}, \qquad
  U^{n}_{-1,j}  = U^{n}_{1,j} - 2 h_x \left( p_j^n U^n_{0,j} + q_j^{n} \right)
\end{equation*}
Подставим в уравнение
\begin{multline*}
  - a_x U^{n+1}_{1,j} + \left( 1 + b_x  \right) U^{n+1}_{0,j} - c_x \left( U^{n+1}_{1,j} - 2 h_x (p_j^{n+1} U^{n+1}_{0,j} + q_j^{n+1}) \right) = \\
  = a_x U^{n}_{1,j} + \left(1 - b_x \right) U^{n}_{0,j} + c_x \left(  U^{n}_{1,j} - 2 h_x (p^n U^{n}_{0,j} + q_j^{n}) \right).
\end{multline*}
\begin{multline*}
  - (a_x + c_x) U^{n+1}_{1,j} + \left( 1 + b_x + 2 c_x h_x p^{n+1}_j \right) U^{n+1}_{0,j} = \\
  = (a_x + c_x) U^{n}_{1,j} + \left(1 - b_x - 2 c_x h_x p^{n} \right) U^{n}_{0,j} - 2 c_x h_x (q_j^{n+1} + q_j^{n}).
\end{multline*}
\begin{equation}
  - b_x U^{n+1}_{1,j} + \left( 1 + b_x  + 2 c_x h_x p^{n+1}_j \right) U^{n+1}_{0,j} =
  b_x U^{n}_{1,j} + \left(1 - b_x - 2 c_x h_x p^{n}_j \right) U^{n}_{0,j} - 2 c_x h_x (q_j^{n+1} + q_j^{n}).
\end{equation}

Обозначим $\alpha_1 = 1 + b_x  + 2 c_x h_x p^{n+1}_j $, $ \beta_1 = -b_x$
\begin{equation}
  \label{eq:2d-bc3-x0}
  \beta_1 U^{n+1}_{1,j} + \alpha_1 U^{n+1}_{0,j} =
  b_x U^{n}_{1,j} + \left(1 - b_x - 2 c_x h_x p^{n}_j \right) U^{n}_{0,j} - 2 c_x h_x (q_j^{n+1} + q_j^{n}).
\end{equation}

Аналогично для границы $x = x_1$:
\begin{equation*}
  \frac{U^{n}_{N_x+1,j} -  U^{n}_{N_x-1,j}}{2 h_x} + p_j^n U^n_{N_x,j}  = q_j^{n}, \qquad
  U^{n}_{N_x+1,j}  = U^{n}_{N_x-1,j} - 2 h_x \left( p_j^n U^n_{N_x,j} - q_j^{n} \right)
\end{equation*}
\begin{multline*}
  - a_x \left( U^{n+1}_{N_x-1,j} - 2 h_x \left( p_j^{n+1} U^{n+1}_{N_x,j} - q_j^{n+1} \right) \right) + \left( 1 + b_x \right) U^{n+1}_{N_x,j} - c_x U^{n+1}_{N_x-1,j} = \\
  = a_x \left(U^{n}_{N_x-1,j} - 2 h_x \left( p_j^n U^n_{N_x,j} - q_j^{n} \right) \right) + \left(1 - b_x  \right) U^{n}_{N_x,j} + c_x U^{n}_{N_x-1,j}.
\end{multline*}
\begin{equation*}
  \left( 1 + b_x + 2 h_x a_x p^{n+1}_j \right) U^{n+1}_{N_x,j}  - (a_x + c_x) U^{n+1}_{N_x-1,j}  = \left(1 - b_x - 2 h_x a_x p^{n}_j \right) U^{n}_{N_x,j} + (a_x + c_x) U^{n}_{N_x-1,j} + 2 a_x h_x \left( q_j^{n+1} + q_j^{n} \right).
\end{equation*}
\begin{equation}
  \left( 1 + b_x + 2 h_x a_x p^{n+1}_j \right) U^{n+1}_{N_x,j}  - b_x U^{n+1}_{N_x-1,j}  = \left(1 - b_x - 2 h_x a_x p^{n}_j \right) U^{n}_{N_x,j} + b_x U^{n}_{N_x-1,j} + 2 a_x h_x \left( q_j^{n+1} + q_j^{n} \right).
\end{equation}

Обозначим $\gamma_1 = -b_x$, $\delta_1 = 1 + b_x + 2 h_x a_x p^{n+1}_j$
\begin{equation}
  \label{eq:2d-bc3-x1}
  \delta_1 U^{n+1}_{N_x,j} + \gamma_1 U^{n+1}_{N_x-1,j}  = \left(1 - b_x - 2 h_x a_x p^{n}_j \right) U^{n}_{N_x,j} + b_x U^{n}_{N_x-1,j} + 2 a_x h_x \left( q_j^{n+1} + q_j^{n} \right).
\end{equation}

По формуле центральной разности для условия на границе $y = y_0$
\begin{equation*}
  \frac{U^{n}_{i,1} -  U^{n}_{i,-1}}{2 h_y} + p_i^n U^n_{i,0}  = q_i^{n}, \qquad
  U^{n}_{i,-1}  = U^{n}_{i,1} - 2 h_y \left( p_i^n U^n_{i,0} + q_i^{n} \right)
\end{equation*}
Подставим в уравнение
\begin{multline*}
  - a_y U^{n+1}_{i,1} + \left( 1 + b_y  \right) U^{n+1}_{i,0} - c_y \left( U^{n+1}_{i,1} - 2 h_y (p_i^{n+1} U^{n+1}_{i,0} + q_i^{n+1}) \right) = \\
  = a_y U^{n}_{i,1} + \left(1 - b_y \right) U^{n}_{i,0} + c_y \left(  U^{n}_{i,1} - 2 h_y (p^n U^{n}_{i,0} + q_i^{n}) \right).
\end{multline*}
\begin{multline*}
  - (a_y + c_y) U^{n+1}_{i,1} + \left( 1 + b_y + 2 c_y h_y p^{n+1}_i \right) U^{n+1}_{i,0} = \\
  = (a_y + c_y) U^{n}_{i,1} + \left(1 - b_y - 2 c_y h_y p^{n}_i \right) U^{n}_{i,0} - 2 c_y h_y (q_i^{n+1} + q_i^{n}).
\end{multline*}
\begin{equation}
  - b_y U^{n+1}_{i,1} + \left( 1 + b_y  + 2 c_y h_y p^{n+1}_i \right) U^{n+1}_{i,0} =
  b_y U^{n}_{i,1} + \left(1 - b_y - 2 c_y h_y p^{n}_i \right) U^{n}_{i,0} - 2 c_y h_y (q_i^{n+1} + q_i^{n}).
\end{equation}

Обозначим $\alpha_2 = 1 + b_y  + 2 c_y h_y p^{n+1}_i $, $ \beta_2 = - b_y$
\begin{equation}
  \label{eq:2d-bc3-y0}
  \beta_2 U^{n+1}_{i,1} + \alpha_2 U^{n+1}_{i,0} =
  b_y U^{n}_{i,1} + \left(1 - b_y - 2 c_y h_y p^{n}_i \right) U^{n}_{i,0} - 2 c_y h_y (q_i^{n+1} + q_i^{n}).
\end{equation}


Аналогично для границы $y = y_1$:
\begin{equation*}
  \frac{U^{n}_{i,N_y+1} -  U^{n}_{i,N_y-1}}{2 h_y} + p^{n}_i U^n_{i,N_y}  = q_i^{n}, \qquad
  U^{n}_{i,N_y+1}  = U^{n}_{i,N_y-1} - 2 h_y \left( p_i^n U^n_{i,N_y} - q_i^{n} \right)
\end{equation*}
\begin{multline*}
  - a_y \left( U^{n+1}_{i,N_y-1} - 2 h_y \left( p_i^{n+1} U^{n+1}_{i,N_y} - q_i^{n+1} \right) \right) + \left( 1 + b_y \right) U^{n+1}_{i,N_y} - c_y U^{n+1}_{i,N_y-1} = \\
  = a_y \left(U^{n}_{i,N_y-1} - 2 h_y \left( p_i^n U^n_{i,N_y} - q_i^{n} \right) \right) + \left(1 - b_y  \right) U^{n}_{i,N_y} + c_y U^{n}_{i,N_y-1}.
\end{multline*}
\begin{equation*}
  \left( 1 + b_y + 2 h_y a_y p^{n+1}_i \right) U^{n+1}_{i,N_y}  - (a_y + c_y) U^{n+1}_{i,N_y-1}  = \left(1 - b_y - 2 h_y a_y p^{n}_i \right) U^{n}_{i,N_y} + (a_y + c_y) U^{n}_{i,N_y-1} + 2 a_y h_y \left( q_i^{n+1} + q_i^{n} \right).
\end{equation*}
\begin{equation}
  \left( 1 + b_y + 2 h_y a_y p^{n+1}_i \right) U^{n+1}_{i,N_y}  - b_y U^{n+1}_{i,N_y-1}  = \left(1 - b_y - 2 h_y a_y p^{n}_i \right) U^{n}_{i,N_y} + b_y U^{n}_{i,N_y-1} + 2 a_y h_y \left( q_i^{n+1} + q_i^{n} \right).
\end{equation}

Обозначим $\gamma_2 = -b_y$, $\delta_2 = 1 + b_y + 2 h_y a_y p^{n+1}_i$
\begin{equation}
  \label{eq:2d-bc3-y1}
  \delta_2 U^{n+1}_{i,N_y} + \gamma_2 U^{n+1}_{i,N_y-1}  = \left(1 - b_y - 2 h_y a_y p^{n}_i \right) U^{n}_{i,N_y} + b_y U^{n}_{i,N_y-1} + 2 a_y h_y \left( q_i^{n+1} + q_i^{n} \right).
\end{equation}


\hrulefill

Объединяя формулы (\ref{eq:2d-scheme}) -- (\ref{eq:2d-bc3-y1}) заключаем, что задача (\ref{eq:2d-problem-eq}) -- (\ref{eq:2d-problem-bc3}) сводится к следующему виду:

\begin{align*}
  - a_x U^*_{i+1,j} + \left(1 + b_x \right) U^*_{i,j} - c_x  U^*_{i-1,j} &=
                                                                           a_x U^n_{i+1,j} + c_x U^n_{i-1,j} + 2  a_y U^n_{i,j+1} + 2 c_y U^n_{i,j-1} +
                                                                           (1 -  b_x - 2 b_y) U^n_{i,j}, \\
  - a_y U^{n+1}_{i,j+1} + \left(1 + b_y \right)  U^{n+1}_{i,j} - c_y U^{n+1}_{i,j-1} &=
                                                                                       U_{i,j}^* - a_y U^n_{i,j+1} - c_y U^n_{i,j-1} + b_y U^n_{i,j},
\end{align*}
где $i = \overline {1, N_x-1}, \ j = \overline {1, N_y-1}, \ n = \overline{0,T}; \ a_x = \frac{\tau D^+_x}{2h^2}, c_x = \frac{\tau D^-_x}{2h^2}, b_x = a_x + c_x \ a_y = \frac{\tau D^+_y}{2h^2}, c_y = \frac{\tau D^-_y}{2h^2}, b_y = a_y + c_y$.

Начальные условия
\begin{equation*}
  U^0_{ij} = \psi_{ij}, \quad i = \overline {1, N_x-1}, \ j = \overline {1, N_y-1}.
\end{equation*}

Примеры граничных условий первого рода на границах $x = x_0$, $x = x_1$, $y = y_0$ и $y = y_1$ соответственно:
\begin{equation*}
  U^n_{0j} = \varphi_j^n, \qquad U^n_{N_x j} = \varphi_j^n, \qquad  U^n_{i0} = \varphi_i^n, \qquad U^n_{i N_y} = \varphi_i^n,
\end{equation*}
где $j = \overline {1, N_y-1}, \ n = \overline{0,T}$.

Примеры граничных условий третьего рода на границах $x = x_0$, $x = x_1$, $y = y_0$ и $y = y_1$ соответственно:
\begin{align*}
  \beta_1 U^{n+1}_{1,j} + \alpha_1 U^{n+1}_{0,j} &=
  b_x U^{n}_{1,j} + \left(1 - b_x - 2 c_x h_x p^{n}_j \right) U^{n}_{0,j} - 2 c_x h_x (q_j^{n+1} + q_j^{n}). \\
  \delta_1 U^{n+1}_{N_x,j} + \gamma_1 U^{n+1}_{N_x-1,j}  &= \left(1 - b_x - 2 h_x a_x p^{n}_j \right) U^{n}_{N_x,j} + b_x U^{n}_{N_x-1,j} + 2 a_x h_x \left( q_j^{n+1} + q_j^{n} \right). \\
  \beta_2 U^{n+1}_{i,1} + \alpha_2 U^{n+1}_{i,0} &=
  b_y U^{n}_{i,1} + \left(1 - b_y - 2 c_y h_y p^{n}_i \right) U^{n}_{i,0} - 2 c_y h_y (q_i^{n+1} + q_i^{n}). \\
  \delta_2 U^{n+1}_{i,N_y} + \gamma_2 U^{n+1}_{i,N_y-1} &= \left(1 - b_y - 2 h_y a_y p^{n}_i \right) U^{n}_{i,N_y} + b_y U^{n}_{i,N_y-1} + 2 a_y h_y \left( q_i^{n+1} + q_i^{n} \right).
\end{align*}
где $n = \overline{0,T}$, $\alpha_1 =  1 + b_x  + 2 c_x h_x p^{n+1}_j$, $ \beta_1 = -b_x$, $\gamma_1 = -b_x$, $\delta_1 = 1 + b_x + 2 h_x a_x p^{n+1}_j$, $\alpha_2 = 1 + b_y  + 2 c_y h_y p^{n+1}_i $, $ \beta_2 = - b_y$, $\gamma_2 = -b_y$, $\delta_2 = 1 + b_y + 2 h_y a_y p^{n+1}_i$

\hrulefill

\section{Схема Дугласа -- Ганна}
\label{sec:heat3d}

В трёхменрном случае задача приобретает вид:
\begin{align}
  \label{eq:3d-problem-eq}
  \frac{\partial U(t,x,y,z)}{\partial t} = \frac{\partial^2}{\partial x^2} &U(t,x,y,z) + \frac{\partial^2}{\partial y^2} U(t,x,y,z) + \frac{\partial^2}{\partial z^2} U(t,x,y,z) + H(t,x,y,z),\\
                                         &x \in \Omega = [x_0,x_1] \times [y_0, y_1] \times [z_0, z_1], \ \partial \Omega = \partial \Omega_1 \cup \partial \Omega_2 \\
  \label{eq:3d-problem-ic}
  U(0, x, y, z) &= g(x, y, z),\\
  \label{eq:3d-problem-bc1}
  \left. U(t,x,y,z) \right|_{(x,y,z) \in \partial \Omega_1} &=  \varphi(t, x,y,z),   \\
  \label{eq:3d-problem-bc3}
  \left. \frac{\partial U(t,x,y,z) }{\partial \vec n} \right|_{(x,y,z) \in \partial \Omega_2}& +
                                                                                               p(t)\left. U(t,x,y) \right|_{(x,y,z) \in \partial \Omega_2}  = \varphi(t, x, y).
\end{align}

Многошаговая реализация метода Дугласа -- Ганна имеет вид:
\begin{equation*}
  \begin{cases}
    \frac{U^{*}   - U^n}{\tau} &= \frac{1}{2} L_1 U^{*}  + \frac{1}{2} L_1 U^n +  L_2 U^n +  L_3 U^n + H^n\\
    \frac{U^{**}  - U^n}{\tau} &= \frac{1}{2} L_1 U^{*}  + \frac{1}{2} L_1 U^n +
    \frac{1}{2} L_2 U^{**} + \frac{1}{2} L_2 U^n + L_3 U^n \\
    \frac{U^{n+1} - U^n}{\tau} &= \frac{1}{2} L_1 U^{*}   + \frac{1}{2} L_1 U^n +
    \frac{1}{2} L_2 U^{**}  + \frac{1}{2} L_2 U^n +
    \frac{1}{2} L_3 U^{n+1} + \frac{1}{2} L_3 U^n +
    \frac{1}{2} \left( H^{n+1} + H^{n} \right)
  \end{cases}
\end{equation*}

Каждое из уравнений системы --- аппроксимация полного уравнения диффузии => не нужно модификаций для граничных условий.
Проведём разделение известных и неизвестных переменных для каждого из уравнений.

\begin{equation*}
  \begin{cases}
    \frac{U^{*}}{\tau}  - \frac{1}{2} L_1 U^{*} &= \frac{U^n}{\tau} + \frac{1}{2} L_1 U^n +  L_2 U^n +  L_3 U^n \\
    \frac{U^{**}}{\tau} - \frac{1}{2} L_2 U^{**} &= \frac{U^n}{\tau} + \frac{1}{2} L_1 U^{*}  + \frac{1}{2} L_1 U^n + \frac{1}{2} L_2 U^n + L_3 U^n \\
    \frac{U^{n+1}}{\tau} - \frac{1}{2} L_3 U^{n+1} &= \frac{U^n}{\tau} + \frac{1}{2} L_1 U^{*}   + \frac{1}{2} L_1 U^n + \frac{1}{2} L_2 U^{**}  + \frac{1}{2} L_2 U^n + \frac{1}{2} L_3 U^n
  \end{cases}
\end{equation*}

Сгруппируем операторы
\begin{equation*}
  \begin{cases}
    \left( \frac{1}{\tau} I - \frac{1}{2} L_1 \right) U^{*} &=
    \left( \frac{1}{\tau} I + \frac{1}{2} L_1 +  L_2 + L_3 \right) U^n \\
    \left( \frac{1}{\tau} I - \frac{1}{2} L_2 \right) U^{**} &=
    \left( \frac{1}{\tau} I + \frac{1}{2} L_1 + \frac{1}{2} L_2 + L_3 \right) U^n + \frac{1}{2} L_1 U^{*}\\
    \left( \frac{1}{\tau} I - \frac{1}{2} L_3 \right) U^{n+1} &=
    \left( \frac{1}{\tau} I  + \frac{1}{2} L_1 + \frac{1}{2} L_2 + \frac{1}{2} L_3 \right) U^n + \frac{1}{2} L_1 U^{*}   + \frac{1}{2} L_2 U^{**}
  \end{cases}
\end{equation*}

Вычтем первое уравнение из второго и второе из третьего:

\begin{equation*}
  \begin{cases}
    &\left( \frac{1}{\tau} I - \frac{1}{2} L_1 \right) U^{*} =
    \left( \frac{1}{\tau} I + \frac{1}{2} L_1 +  L_2 + L_3 \right) U^n \\
    &\left( \frac{1}{\tau} I - \frac{1}{2} L_2 \right) U^{**} -
    \left( \frac{1}{\tau} I - \frac{1}{2} L_1 \right) U^{*} = \\
    & \qquad = \left( \frac{1}{\tau} I + \frac{1}{2} L_1 + \frac{1}{2} L_2 + L_3 \right) U^n + \frac{1}{2} L_1 U^{*} - \\
    & \qquad - \left( \frac{1}{\tau} I + \frac{1}{2} L_1 +  L_2 + L_3 \right) U^n \\
    &\left( \frac{1}{\tau} I - \frac{1}{2} L_3 \right) U^{n+1} -
    \left( \frac{1}{\tau} I - \frac{1}{2} L_2 \right) U^{**} = \\
    & \qquad = \left( \frac{1}{\tau} I + \frac{1}{2} L_1 + \frac{1}{2} L_2 + \frac{1}{2} L_3 \right) U^n + \frac{1}{2} L_1 U^{*}   + \frac{1}{2} L_2 U^{**} - \\
    & \qquad - \left( \frac{1}{\tau} I + \frac{1}{2} L_1 + \frac{1}{2} L_2 + L_3 \right) U^n - \frac{1}{2} L_1 U^{*}
  \end{cases}
\end{equation*}

Отсюда
\begin{equation*}
  \begin{cases}
    \left( \frac{1}{\tau} I - \frac{1}{2} L_1 \right) U^{*} &=
    \left( \frac{1}{\tau} I + \frac{1}{2} L_1 +  L_2 + L_3 \right) U^n \\
    \left( \frac{1}{\tau} I - \frac{1}{2} L_2 \right) U^{**} -
    \left( \frac{1}{\tau} I - \frac{1}{2} L_1 \right) U^{*} &= - \frac{1}{2} L_2 U^n + \frac{1}{2} L_1 U^{*} \\
    \left( \frac{1}{\tau} I - \frac{1}{2} L_3 \right) U^{n+1} -
    \left( \frac{1}{\tau} I - \frac{1}{2} L_2 \right) U^{**} &= - \frac{1}{2} L_3 U^n + \frac{1}{2} L_2 U^{**}
  \end{cases}
\end{equation*}

Переносим вычитаемые в правую часть и сокращаем
\begin{equation*}
  \begin{cases}
    \left( \frac{1}{\tau} I - \frac{1}{2} L_1 \right) U^{*} &=
    \left( \frac{1}{\tau} I + \frac{1}{2} L_1 +  L_2 + L_3 \right) U^n \\
    \left( \frac{1}{\tau} I - \frac{1}{2} L_2 \right) U^{**} & =
    \frac{1}{\tau} U^{*} - \frac{1}{2} L_2 U^n \\
    \left( \frac{1}{\tau} I - \frac{1}{2} L_3 \right) U^{n+1} & =
    \frac{1}{\tau} U^{**} - \frac{1}{2} L_3 U^n
  \end{cases}
\end{equation*}

Домножим все части на $\tau$:
\begin{equation*}
  \label{eq:system}
  \begin{cases}
    \left( I - \frac{\tau}{2} L_1 \right) U^{*} &=
    \left( I + \frac{\tau}{2} L_1 +  \tau L_2 + \tau L_3 \right) U^n \\
    \left( I - \frac{\tau}{2} L_2 \right) U^{**} & = U^{*} - \frac{\tau}{2} L_2 U^n \\
    \left( I - \frac{\tau}{2} L_3 \right) U^{n+1} & =  U^{**} - \frac{\tau}{2} L_3 U^n
  \end{cases}
\end{equation*}

Операторы $L_1$, $L_2$ и $L_3$ имеют вид:
\begin{equation*}
  \begin{aligned}
    L_1 U &= \frac{D^+_x \frac{U_{i+1,j,k} - U_{i,j,k}}{h} - D^-_x\frac{U_{i,j,k} - U_{i-1,j,k}}{h}}{h}
    = \frac{D^+_x}{h^2}U_{i+1,j,k} - \frac{D^+_x}{h^2}U_{i,j,k} - \frac{D^-_x}{h^2} U_{i,j,k} + \frac{D^-_x}{h^2}U_{i-1,j,k} , \quad \forall i,j,k; \\
    L_2 U &= \frac{D^+_y}{h^2}U_{i,j+1,k} - \frac{D^+_y}{h^2}U_{i,j,k} - \frac{D^-_y}{h^2} U_{i,j,k} + \frac{D^-_y}{h^2}U_{i,j-1,k} , \quad \forall i,j,k;\\
    L_3 U &= \frac{D^+_z}{h^2}U_{i,j,k+1} - \frac{D^+_z}{h^2}U_{i,j,k} - \frac{D^-_z}{h^2} U_{i,j,k} + \frac{D^-_z}{h^2}U_{i,j,k-1} , \quad \forall i,j,k;
  \end{aligned}
\end{equation*}

Перепишем левые части уравнений системы (\ref{eq:system}) в виде, удобном для заполнения матриц
\begin{equation*}
  \begin{aligned}
    \left( I - \frac{\tau}{2} L_1 \right) U^{*} = U^*_{i,j,k} - \frac{\tau}{2} \left[ \frac{D^+_x}{h^2}U^*_{i+1,j,k} - \frac{D^+_x}{h^2}U^*_{i,j,k} - \frac{D^-_x}{h^2} U^*_{i,j,k} + \frac{D^-_x}{h^2}U^*_{i-1,j,k} \right] = \\
    = - \frac{\tau D^+_x}{2h^2} U^{*}_{i+1,j,k} + \left( 1 + \frac{\tau D^+_x}{2h^2} + \frac{\tau D^-_x}{2h^2} \right) U^{*}_{i,j,k} - \frac{\tau D^-_x}{2 h^2} U^{*}_{i-1,j,k}, \\
    \left( I - \frac{\tau}{2} L_2 \right) U^{**} = - \frac{\tau D^+_y}{2h^2} U^{**}_{i,j+1,k} + \left( 1  + \frac{\tau D^+_y}{2h^2} + \frac{\tau D^-_y}{2h^2} \right) U^{**}_{i,j,k} - \frac{\tau D^-_y}{2 h^2} U^{**}_{i,j-1,k}\\
    \left( I - \frac{\tau}{2} L_3 \right) U^{n+1} = - \frac{\tau D^+_z}{2h^2} U^{n+1}_{i,j,k+1} + \left( 1 + \frac{\tau D^+_z}{2h^2} + \frac{\tau D^-_z}{2h^2} \right) U^{n+1}_{i,j,k} - \frac{\tau D^-_z}{2 h^2} U^{n+1}_{i,j,k-1}
  \end{aligned}
\end{equation*}

Обозначим  $a_x = \frac{\tau D^+_x}{2h^2}, c_x = \frac{\tau D^-_x}{2h^2}$, $a_y = \frac{\tau D^+_y}{2h^2}, c_y = \frac{\tau D^-_y}{2h^2}$ и $a_z = \frac{\tau D^+_z}{2h^2}, c_z = \frac{\tau D^-_z}{2h^2}$ соответственно. Тогда
\begin{equation*}
  \begin{aligned}
    \left( I - \frac{\tau}{2} L_1 \right) U^{*} = - a_x U^{*}_{i+1,j,k} + \left( 1 + a_x + c_x \right) U^{*}_{i,j,k} - c_x U^{*}_{i-1,j,k}, \\
    \left( I - \frac{\tau}{2} L_2 \right) U^{**} = - a_y U^{**}_{i,j+1,k} + \left( 1  + a_y + c_y \right) U^{**}_{i,j,k} - c_y U^{**}_{i,j-1,k}\\
    \left( I - \frac{\tau}{2} L_3 \right) U^{n+1} = - a_z U^{n+1}_{i,j,k+1} + \left( 1 + a_z + c_z \right) U^{n+1}_{i,j,k} - c_z U^{n+1}_{i,j,k-1}
  \end{aligned}
\end{equation*}

Распишем правую часть первого уравнения системы (\ref{eq:system})
\begin{multline*}
  \left( I + \frac{\tau}{2} L_1 +  \tau L_2 + \tau L_3 \right) U
  = U_{i,j,k} + \frac{\tau}{2} \left[\frac{D^+_x}{h^2}U_{i+1,j,k} - \frac{D^+_x}{h^2}U_{i,j,k} - \frac{D^-_x}{h^2} U_{i,j,k} + \frac{D^-_x}{h^2}U_{i-1,j,k} \right] +  \\
  + \tau \left[\frac{D^+_y}{h^2}U_{i,j+1,k} - \frac{D^+_y}{h^2}U_{i,j,k} - \frac{D^-_y}{h^2} U_{i,j,k} + \frac{D^-_y}{h^2}U_{i,j-1,k}  \right] + \\
  + \tau \left[ \frac{D^+_z}{h^2}U_{i,j,k+1} - \frac{D^+_z}{h^2}U_{i,j,k} - \frac{D^-_z}{h^2} U_{i,j,k} + \frac{D^-_z}{h^2}U_{i,j,k-1} \right] = \\
  = \frac{\tau D^+_x}{2h^2} U_{i+1,j,k} + \frac{\tau D^-_x}{2h^2} U_{i-1,j,k} + \frac{\tau D^+_y}{h^2} U_{i,j+1,k} + \frac{\tau D^-_y}{h^2} U_{i,j-1,k} + \frac{\tau D^+_z}{h^2} U_{i,j,k+1} + \frac{\tau D^-_z}{h^2} U_{i,j,k-1} + \\
  + \left[ 1 - \frac{\tau D^+_x}{2h^2} - \frac{\tau D^-_x}{2h^2} - \frac{\tau D^+_y}{h^2} - \frac{\tau D^-_y}{h^2} - \frac{\tau D^+_z}{h^2} - \frac{\tau D^-_z}{h^2} \right] U_{i,j,k} = \\
  = a_x U_{i+1,j,k} + c_x U_{i-1,j,k} + 2a_y U_{i,j+1,k} + 2c_y U_{i,j-1,k} + 2a_z U_{i,j,k+1} + 2c_z U_{i,j,k-1} + \\
  + \left[ 1 - a_x - c_x - 2a_y - 2c_y - 2a_z - 2c_z \right] U_{i,j,k} , \quad \forall i,j,k; \\
\end{multline*}

То же для остальных уравнений:
\begin{multline*}
  U^{*} - \frac{\tau}{2} L_2 U = U^{*}_{i,j,k} - \frac{\tau}{2} \left[ \frac{D^+_y}{h^2}U_{i,j+1,k} - \frac{D^+_y}{h^2}U_{i,j,k} - \frac{D^-_y}{h^2} U_{i,j,k} + \frac{D^-_y}{h^2}U_{i,j-1,k} \right] = \\
  = U^{*}_{i,j,k} - a_y U_{i,j+1,k} + \left( a_y + c_y \right) U_{i,j,k}  - c_y U_{i,j-1,k} = \\
  = U^{*}_{i,j,k} - a_y U_{i,j+1,k} + b_y U_{i,j,k}  - c_y U_{i,j-1,k}
\end{multline*}
\begin{multline*}
  U^{**} - \frac{\tau}{2} L_3 U = U^{**}_{i,j,k} - \frac{\tau}{2} \left[ \frac{D^+_z}{h^2}U_{i,j,k+1} - \frac{D^+_z}{h^2}U_{i,j,k} - \frac{D^-_z}{h^2} U_{i,j,k} + \frac{D^-_z}{h^2}U_{i,j,k-1} \right] = \\
  = U^{**}_{i,j,k} - a_z U_{i,j,k+1}  + \left( a_z + c_z \right) U_{i,j,k} - c_z U_{i,j,k-1} = \\
  = U^{**}_{i,j,k} - a_z U_{i,j,k+1}  + b_z U_{i,j,k} - c_z U_{i,j,k-1}
\end{multline*}


Обозначая $b_x = a_x + c_x$, $b_y = a_y + c_y$ и $b_z = a_z + c_z$

Окончательно на $n$-ом шаге имеем:
\begin{multline}
  \label{eq:3d-scheme}
  - a_x U^{*}_{i+1,j,k} + \left( 1 + b_x \right) U^{*}_{i,j,k} - c_x U^{*}_{i-1,j,k}
  = a_x U_{i+1,j,k} + c_x U_{i-1,j,k} + 2a_y U_{i,j+1,k}  + 2c_y U_{i,j-1,k} + \\ + 2a_z U_{i,j,k+1} + 2c_z U_{i,j,k-1}
  + \left[ 1 - b_x - 2b_y - 2b_z \right] U_{i,j,k}
\end{multline}
\begin{equation*}
  - a_y U^{**}_{i,j+1,k} + \left( 1  + b_y \right) U^{**}_{i,j,k} - c_y U^{**}_{i,j-1,k} =
  U^{*}_{i,j,k} - a_y U_{i,j+1,k} + b_y U_{i,j,k} - c_y U_{i,j-1,k}
\end{equation*}
\begin{equation*}
  - a_z U^{n+1}_{i,j,k+1} + \left( 1 + b_z \right) U^{n+1}_{i,j,k} - c_z U^{n+1}_{i,j,k-1} =  U^{**}_{i,j,k} - a_z U_{i,j,k+1}  + b_z U_{i,j,k} - c_z U_{i,j,k-1}
\end{equation*}
где $a_x = \frac{\tau D^+_x}{2h^2}, c_x = \frac{\tau D^-_x}{2h^2}, b_x = a_x + c_x$, $a_y = \frac{\tau D^+_y}{2h^2}, c_y = \frac{\tau D^-_y}{2h^2}, b_y = a_y + c_y$ и $a_z = \frac{\tau D^+_z}{2h^2}, c_z = \frac{\tau D^-_z}{2h^2}, b_z = a_z + c_z$ соответственно.

Начальные условия
\begin{equation}
  \label{eq:3d-ic}
  U^0_{ijk} = \psi_{ijk}, \quad i = \overline {1, N_x-1}, \ j = \overline {1, N_y-1}, \ k = \overline {1, N_z-1}.
\end{equation}

Примеры граничных условий первого рода:
\begin{equation*}
  \label{eq:3d-bc1}
  U^n_{0jk} = \varphi_{jk}^n, \qquad U^n_{N_x j k} = \varphi_{jk}^n, \qquad  U^n_{i0k} = \varphi_{ik}^n, \qquad U^n_{i N_y k} = \varphi_{ik}^n, \qquad  U^n_{ij0} = \varphi_{ij}^n, \qquad U^n_{i j N_z} = \varphi_{ij}^n,
\end{equation*}
где $i = \overline {1, N_x-1}, \ j = \overline {1, N_y-1}, \ k = \overline {1, N_z-1}, \ n = \overline{0,T}$.

Рассмотрим примеры граничных условий третьего рода:
\begin{align*}
  &\left. \frac{ \partial U(t,x,y,z) }{\partial x} \right|_{x = x_0} +
    p(t) U(t, x_0, y, z) = q(t, y, z),
    \left. \frac{ \partial U(t,x,y,z) }{\partial x} \right|_{x = x_1} +
    p(t) U(t, x_1, y, z) = q(t, y, z), \\
  &\left. \frac{ \partial U(t,x,y,z) }{\partial y} \right|_{y = y_0} +
    p(t) U(t, x, y_0,z)= q(t, x,z),
    \left. \frac{ \partial U(t,x,y,z) }{\partial y} \right|_{y = y_1} +
    p(t) U(t, x, y_1,z)= q(t, x,z),  \\
  &\left. \frac{ \partial U(t,x,y,z) }{\partial z} \right|_{z = z_0} +
    p(t) U(t, x, y, z_0)= q(t, x, y),
    \left. \frac{ \partial U(t,x,y,z) }{\partial z} \right|_{z = z_1} +
    p(t) U(t, x, y, z_1)= q(t, x, y),
\end{align*}
где $x \in [x_0, x_1], y \in [y_0,y_1], z \in [z_0,z_1], t \in [t_0, t_1]$.


По формуле центральной разности для условия на границе $x = x_0$
\begin{equation*}
  \frac{U^{n}_{1,j,k} -  U^{n}_{-1,j,k}}{2 h_x} + p_j^n U^n_{0,j,k}  = q_{jk}^{n}, \qquad
  U^{n}_{-1,j,k}  = U^{n}_{1,j,k} - 2 h_x \left( p_{jk}^n U^n_{0,j,k} + q_{jk}^{n} \right)
\end{equation*}
Подставим в уравнение
\begin{multline*}
  - a_x U^{n+1}_{1,j,k} + \left( 1 + b_x  \right) U^{n+1}_{0,j,k} - c_x \left( U^{n+1}_{1,j,k} - 2 h_x (p^{n+1}_{jk} U^{n+1}_{0,j,k} + q_{jk}^{n+1}) \right) = \\
  = a_x U^{n}_{1,j,k} + \left(1 - b_x \right) U^{n}_{0,j,k} + c_x \left(  U^{n}_{1,j,k} - 2 h_x (p^n_{jk} U^{n}_{0,j,k} + q_{jk}^{n}) \right).
\end{multline*}
\begin{multline*}
  - (a_x + c_x) U^{n+1}_{1,j,k} + \left( 1 + b_x + 2 c_x h_x p^{n+1}_{jk} \right) U^{n+1}_{0,j,k} = \\
  = (a_x + c_x) U^{n}_{1,j,k} + \left(1 - b_x - 2 c_x h_x p^{n}_{jk} \right) U^{n}_{0,j,k} - 2 c_x h_x (q_{jk}^{n+1} + q_{jk}^{n}).
\end{multline*}
\begin{equation}
  - b_x U^{n+1}_{1,j,k} + \left( 1 + b_x  + 2 c_x h_x p^{n+1}_{jk} \right) U^{n+1}_{0,j,k} =
  b_x U^{n}_{1,j,k} + \left(1 - b_x - 2 c_x h_x p^{n}_{jk} \right) U^{n}_{0,j,k} - 2 c_x h_x (q_{jk}^{n+1} + q_{jk}^{n}).
\end{equation}

Обозначим $\alpha_1 = 1 + b_x  + 2 c_x h_x p^{n+1}_{jk} $, $ \beta_1 = - b_x$
\begin{equation}
  \label{eq:3d-bc3-x0}
  \beta_1 U^{n+1}_{1,j,k} + \alpha_1 U^{n+1}_{0,j,k} =
  b_x U^{n}_{1,j,k} + \left(1 - b_x - 2 c_x h_x p^{n}_{jk} \right) U^{n}_{0,j,k} - 2 c_x h_x (q_{jk}^{n+1} + q_{jk}^{n}).
\end{equation}


Аналогично для границы $x = x_1$:
\begin{equation*}
  \frac{U^{n}_{N_x+1,j,k} -  U^{n}_{N_x-1,j,k}}{2 h_x} + p_{jk}^n U^n_{N_x,j,k}  = q_{jk}^{n}, \qquad
  U^{n}_{N_x+1,j,k}  = U^{n}_{N_x-1,j,k} - 2 h_x \left( p_{jk}^n U^n_{N_x,j,k} - q_{jk}^{n} \right)
\end{equation*}
\begin{multline*}
  - a_x \left( U^{n+1}_{N_x-1,j,k} - 2 h_x \left( p_{jk}^{n+1} U^{n+1}_{N_x,j,k} - q_{jk}^{n+1} \right) \right) + \left( 1 + b_x \right) U^{n+1}_{N_x,j,k} - c_x U^{n+1}_{N_x-1,j,k} = \\
  = a_x \left(U^{n}_{N_x-1,j,k} - 2 h_x \left( p_{kj}^n U^n_{N_x,j,k} - q_{jk}^{n} \right) \right) + \left(1 - b_x  \right) U^{n}_{N_x,j,k} + c_x U^{n}_{N_x-1,j,k}.
\end{multline*}
\begin{equation*}
  \left( 1 + b_x + 2 h_x a_x p^{n+1}_{jk} \right) U^{n+1}_{N_x,j,k}  - (a_x + c_x) U^{n+1}_{N_x-1,j,k}  = \left(1 - b_x - 2 h_x a_x p^{n}_{jk} \right) U^{n}_{N_x,j,k} + (a_x + c_x) U^{n}_{N_x-1,j,k} + 2 a_x h_x \left( q_{jk}^{n+1} + q_{jk}^{n} \right).
\end{equation*}
\begin{equation}
  \left( 1 + b_x + 2 h_x a_x p^{n+1}_{jk} \right) U^{n+1}_{N_x,j,k}  - b_x U^{n+1}_{N_x-1,j,k}  = \left(1 - b_x - 2 h_x a_x p^{n}_{jk} \right) U^{n}_{N_x,j,k} + b_x U^{n}_{N_x-1,j,k} + 2 a_x h_x \left( q_{jk}^{n+1} + q_{jk}^{n} \right).
\end{equation}

Обозначим $\gamma_1 = -b_x$, $\delta_1 = 1 + b_x + 2 h_x a_x p^{n+1}_{jk}$
\begin{equation}
  \label{eq:3d-bc3-x1}
  \delta_1 U^{n+1}_{N_x,j,k} + \gamma_1 U^{n+1}_{N_x-1,j,k}  = \left(1 - b_x - 2 h_x a_x p^{n}_{jk} \right) U^{n}_{N_x,j,k} + b_x U^{n}_{N_x-1,j,k} + 2 a_x h_x \left( q_{jk}^{n+1} + q_{jk}^{n} \right).
\end{equation}


По формуле центральной разности для условия на границе $y = y_0$
\begin{equation*}
  \frac{U^{n}_{i,1,k} -  U^{n}_{i,-1,k}}{2 h_y} + p_{ik}^n U^n_{i,0,k}  = q_{ik}^{n}, \qquad
  U^{n}_{i,-1,k}  = U^{n}_{i,1,k} - 2 h_y \left( p_{ik}^n U^n_{i,0,k} + q_{ik}^{n} \right)
\end{equation*}
Подставим в уравнение
\begin{multline*}
  - a_y U^{n+1}_{i,1,k} + \left( 1 + b_y \right) U^{n+1}_{i,0,k} - c_y \left( U^{n+1}_{i,1,k} - 2 h_y (p^{n+1}_{ik} U^{n+1}_{i,0,k} + q_{ik}^{n+1}) \right) = \\
  = a_y U^{n}_{i,1,k} + \left(1 - b_y \right) U^{n}_{i,0,k} + c_y \left(  U^{n}_{i,1,k} - 2 h_y (p^{n}_{ik} U^{n}_{i,0,k} + q_{ik}^{n}) \right).
\end{multline*}
\begin{multline*}
  - (a_y + c_y) U^{n+1}_{i,1,k} + \left( 1 + b_y + 2 c_y h_y p^{n+1}_{ik} \right) U^{n+1}_{i,0,k} = \\
  = (a_y + c_y) U^{n}_{i,1,k} + \left(1 - b_y - 2 c_y h_y p^{n+1}_{ik} \right) U^{n}_{i,0,k} - 2 c_y h_y (q_{ik}^{n+1} + q_{ik}^{n}).
\end{multline*}
\begin{equation}
  - b_y U^{n+1}_{i,1,k} + \left( 1 + b_y  + 2 c_y h_y p^{n+1}_{ik} \right) U^{n+1}_{i,0,k} =
  b_y U^{n}_{i,1,k} + \left(1 - b_y - 2 c_y h_y p^{n}_{ik} \right) U^{n}_{i,0,k} - 2 c_y h_y (q_{ik}^{n+1} + q_{ik}^{n}).
\end{equation}

Обозначим $\alpha_2 = 1 + b_y  + 2 c_y h_y p^{n+1}_{ik}$, $ \beta_2 = - b_y$
\begin{equation}
  \label{eq:3d-bc3-y0}
  \beta_2 U^{n+1}_{i,1,k} + \alpha_2 U^{n+1}_{i,0,k} =
  b_y U^{n}_{i,1,k} + \left(1 - b_y - 2 c_y h_y p^{n}_{ik} \right) U^{n}_{i,0,k} - 2 c_y h_y (q_{ik}^{n+1} + q_{ik}^{n}).
\end{equation}





Аналогично для границы $y = y_1$:
\begin{equation*}
  \frac{U^{n}_{i,N_y+1,k} -  U^{n}_{i,N_y-1,k}}{2 h_y} + p^{n}_{ik} U^n_{i,N_y,k}  = q_{ik}^{n}, \qquad
  U^{n}_{i,N_y+1,k}  = U^{n}_{i,N_y-1,k} - 2 h_y \left( p^{n}_{ik} U^n_{i,N_y,k} - q_{ik}^{n} \right)
\end{equation*}
\begin{multline*}
  - a_y \left( U^{n+1}_{i,N_y-1,k} - 2 h_y \left( p^{n+1}_{ik} U^{n+1}_{i,N_y,k} - q_{ik}^{n+1} \right) \right) + \left( 1 + b_y \right) U^{n+1}_{i,N_y,k} - c_y U^{n+1}_{i,N_y-1,k} = \\
  = a_y \left(U^{n}_{i,N_y-1,k} - 2 h_y \left( p^{n}_{ik} U^n_{i,N_y,k} - q^{n}_{ik} \right) \right) + \left(1 - b_y  \right) U^{n}_{i,N_y,k} + c_y U^{n}_{i,N_y-1,k}.
\end{multline*}
\begin{equation*}
  \left( 1 + b_y + 2 h_y a_y p^{n+1}_{ik}\right) U^{n+1}_{i,N_y,k}  - (a_y + c_y) U^{n+1}_{i,N_y-1,k}  = \left(1 - b_y - 2 h_y a_y p^{n}_{ik}\right) U^{n}_{i,N_y,k} + (a_y + c_y) U^{n}_{i,N_y-1,k} + 2 a_y h_y \left( q_{ik}^{n+1} + q_{ik}^{n} \right).
\end{equation*}
\begin{equation}
  \left( 1 + b_y + 2 h_y a_y p^{n+1}_{ik}\right) U^{n+1}_{i,N_y,k}  - b_y U^{n+1}_{i,N_y-1,k}  = \left(1 - b_y - 2 h_y a_yp^{n}_{ik} \right) U^{n}_{i,N_y,k} + b_y U^{n}_{i,N_y-1,k} + 2 a_y h_y \left( q_{ik}^{n+1} + q_{ik}^{n} \right).
\end{equation}

Обозначим $\gamma_2 = -b_y$, $\delta_2 = 1 + b_y + 2 h_y a_y p^{n+1}_{ik}$
\begin{equation}
  \label{eq:3d-bc3-y1}
  \delta_2 U^{n+1}_{i,N_y,k}  + \gamma_2 U^{n+1}_{i,N_y-1,k}  = \left(1 - b_y - 2 h_y a_yp^{n}_{ik} \right) U^{n}_{i,N_y,k} + b_y U^{n}_{i,N_y-1,k} + 2 a_y h_y \left( q_{ik}^{n+1} + q_{ik}^{n} \right).
\end{equation}

По формуле центральной разности для условия на границе $z = z_0$
\begin{equation*}
  \frac{U^{n}_{i,j,1} -  U^{n}_{i,j,-1}}{2 h_z} + p_{ij}^n U^n_{i,j,0}  = q_{ij}^{n}, \qquad
  U^{n}_{i,j,-1}  = U^{n}_{i,j,1} - 2 h_z \left( p_{ij}^n U^n_{i,j,0} + q_{ij}^{n} \right)
\end{equation*}
Подставим в уравнение
\begin{multline*}
  - a_z U^{n+1}_{i,j,1} + \left( 1 + b_z  \right) U^{n+1}_{i,j,0} - c_z \left( U^{n+1}_{i,j,1} - 2 h_z (p^{n+1}_{ij} U^{n+1}_{i,j,0} + q_{ij}^{n+1}) \right) = \\
  = a_z U^{n}_{i,j,1} + \left(1 - b_z \right) U^{n}_{i,j,0} + c_z \left(  U^{n}_{i,j,1} - 2 h_z (p^{n}_{ij} U^{n}_{i,j,0} + q_{ij}^{n}) \right).
\end{multline*}
\begin{multline*}
  - (a_z + c_z) U^{n+1}_{i,j,1} + \left( 1 + b_z + 2 c_z h_z p^{n+1}_{ij} \right) U^{n+1}_{i,j,0} = \\
  = (a_z + c_z) U^{n}_{i,j,1} + \left(1 - b_z - 2 c_z h_z p^{n}_{ij} \right) U^{n}_{i,j,0} - 2 c_z h_z (q_{ij}^{n+1} + q_{ij}^{n}).
\end{multline*}
\begin{equation}
  - b_z U^{n+1}_{i,j,1} + \left( 1 + b_z  + 2 c_z h_z p^{n+1}_{ij} \right) U^{n+1}_{i,j,0} =
  b_z U^{n}_{i,j,1} + \left(1 - b_z - 2 c_z h_z p^{n}_{ij} \right) U^{n}_{i,j,0} - 2 c_z h_z (q_{ij}^{n+1} + q_{ij}^{n}).
\end{equation}

Обозначим $\alpha_3 = 1 + b_z  + 2 c_z h_z p^{n+1}_{ij}$, $ \beta_3 = - b_z$
\begin{equation}
  \label{eq:3d-bc3-z0}
  \beta_3 U^{n+1}_{i,j,1} + \alpha_3 U^{n+1}_{i,j,0} =
  b_z U^{n}_{i,j,1} + \left(1 - b_z - 2 c_z h_z p^{n}_{ij} \right) U^{n}_{i,j,0} - 2 c_z h_z (q_{ij}^{n+1} + q_{ij}^{n}).
\end{equation}

Аналогично для границы $z = z_1$:
\begin{equation*}
  \frac{U^{n}_{i,j,N_z+1} -  U^{n}_{i,j,N_z-1}}{2 h_z} + p^{n}_{ij} U^n_{i,j,N_z}  = q_{ij}^{n}, \qquad
  U^{n}_{i,j,N_z+1}  = U^{n}_{i,j,N_z-1} - 2 h_z \left( p^{n}_{ij} U^n_{i,j,N_z} - q_{ij}^{n} \right)
\end{equation*}
\begin{multline*}
  - a_z \left( U^{n+1}_{i,j,N_z-1} - 2 h_z \left( p_{ij}^{n+1} U^{n+1}_{i,j,N_z} - q_{ij}^{n+1} \right) \right) + \left( 1 + b_z \right) U^{n+1}_{i,j,N_z} - c_z U^{n+1}_{i,j,N_z-1} = \\
  = a_z \left(U^{n}_{i,j,N_z-1} - 2 h_z \left(p^{n}_{ij} U^n_{i,j,N_z} - q_{ij}^{n} \right) \right) + \left(1 - b_z  \right) U^{n}_{i,j,N_z} + c_z U^{n}_{i,j,N_z-1}.
\end{multline*}
\begin{equation*}
  \left( 1 + b_z + 2 h_z a_z p^{n+1}_{ij} \right) U^{n+1}_{i,j,N_z}  - (a_z + c_z) U^{n+1}_{i,j,N_z-1}  = \left(1 - b_z - 2 h_z a_z p^{n}_{ij} \right) U^{n}_{i,j,N_z} + (a_z + c_z) U^{n}_{i,j,N_z-1} + 2 a_z h_z \left( q_{ij}^{n+1} + q_{ij}^{n} \right).
\end{equation*}
\begin{equation}
  \left( 1 + b_z + 2 h_z a_z p^{n+1}_{ij} \right) U^{n+1}_{i,j,N_z}  - b_z U^{n+1}_{i,j,N_z-1}  = \left(1 - b_z - 2 h_z a_z p^{n}_{ij}\right) U^{n}_{i,j,N_z} + b_z U^{n}_{i,j,N_z-1} + 2 a_z h_z \left( q_{ij}^{n+1} + q_{ij}^{n} \right).
\end{equation}

Обозначим $\gamma_3 = - b_z$, $\delta_3 =  1 + b_z + 2 h_z a_z p^{n+1}_{ij}$
\begin{equation}
  \label{eq:3d-bc3-z1}
  \delta_3 U^{n+1}_{i,j,N_z} + \gamma_3 U^{n+1}_{i,j,N_z-1}  = \left(1 - b_z - 2 h_z a_z p^{n}_{ij}\right) U^{n}_{i,j,N_z} + b_z U^{n}_{i,j,N_z-1} + 2 a_z h_z \left( q_{ij}^{n+1} + q_{ij}^{n} \right).
\end{equation}

\hrulefill

Объединяя формулы (\ref{eq:3d-scheme}) -- (\ref{eq:3d-bc3-z1}) заключаем, что задача (\ref{eq:3d-problem-eq}) -- (\ref{eq:3d-problem-bc3}) сводится к следующему виду:

\begin{multline*}
  - a_x U^{*}_{i+1,j,k} + \left( 1 + b_x \right) U^{*}_{i,j,k} - c_x U^{*}_{i-1,j,k}
  = a_x U_{i+1,j,k} + c_x U_{i-1,j,k} + 2a_y U_{i,j+1,k}  + 2c_y U_{i,j-1,k} + \\ + 2a_z U_{i,j,k+1} + 2c_z U_{i,j,k-1}
  + \left[ 1 - b_x - 2b_y - 2b_z \right] U_{i,j,k}
\end{multline*}
\begin{equation*}
  - a_y U^{**}_{i,j+1,k} + \left( 1  + b_y \right) U^{**}_{i,j,k} - c_y U^{**}_{i,j-1,k} =
  U^{*}_{i,j,k} - a_y U_{i,j+1,k} + b_y U_{i,j,k} - c_y U_{i,j-1,k}
\end{equation*}
\begin{equation*}
  - a_z U^{n+1}_{i,j,k+1} + \left( 1 + b_z \right) U^{n+1}_{i,j,k} - c_z U^{n+1}_{i,j,k-1} =  U^{**}_{i,j,k} - a_z U_{i,j,k+1}  + b_z U_{i,j,k} - c_z U_{i,j,k-1}
\end{equation*}
где $a_x = \frac{\tau D^+_x}{2h^2}, c_x = \frac{\tau D^-_x}{2h^2}, b_x = a_x + c_x$, $a_y = \frac{\tau D^+_y}{2h^2}, c_y = \frac{\tau D^-_y}{2h^2}, b_y = a_y + c_y$ и $a_z = \frac{\tau D^+_z}{2h^2}, c_z = \frac{\tau D^-_z}{2h^2}, b_z = a_z + c_z$ соответственно.

Начальные условия
\begin{equation*}
  U^0_{ijk} = \psi_{ijk}, \quad i = \overline {1, N_x-1}, \ j = \overline {1, N_y-1}, \ k = \overline {1, N_z-1}.
\end{equation*}

Примеры граничных условий первого рода на границах $x = x_0$, $x = x_1$, $y = y_0$, $y = y_1$, $z = z_0$ и $z = z_1$ соответственно:
\begin{equation*}
  \label{eq:3d-bc1}
  U^n_{0jk} = \varphi_{jk}^n, \qquad U^n_{N_x j k} = \varphi_{jk}^n, \qquad  U^n_{i0k} = \varphi_{ik}^n, \qquad U^n_{i N_y k} = \varphi_{ik}^n, \qquad  U^n_{ij0} = \varphi_{ij}^n, \qquad U^n_{i j N_z} = \varphi_{ij}^n,
\end{equation*}
где $i = \overline {1, N_x-1}, \ j = \overline {1, N_y-1}, \ k = \overline {1, N_z-1}, \ n = \overline{0,T}$.

\begin{align*}
  \beta_1 U^{n+1}_{1,j,k} + \alpha_1 U^{n+1}_{0,j,k} &=
  b_x U^{n}_{1,j,k} + \left(1 - b_x - 2 c_x h_x p^{n}_{jk} \right) U^{n}_{0,j,k} -
  2 c_x h_x (q_{jk}^{n+1} + q_{jk}^{n}). \\
  \delta_1 U^{n+1}_{N_x,j,k} + \gamma_1 U^{n+1}_{N_x-1,j,k} &=
  \left(1 - b_x - 2 h_x a_x p^{n}_{jk} \right) U^{n}_{N_x,j,k} + b_x U^{n}_{N_x-1,j,k} +
  2 a_x h_x \left( q_{jk}^{n+1} + q_{jk}^{n} \right). \\
  \beta_2 U^{n+1}_{i,1,k} + \alpha_2 U^{n+1}_{i,0,k} &=
  b_y U^{n}_{i,1,k} + \left(1 - b_y - 2 c_y h_y p^{n}_{ik} \right) U^{n}_{i,0,k} -
  2 c_y h_y (q_{ik}^{n+1} + q_{ik}^{n}). \\
  \delta_2 U^{n+1}_{i,N_y,k}  + \gamma_2 U^{n+1}_{i,N_y-1,k} &=
  \left(1 - b_y - 2 h_y a_yp^{n}_{ik} \right) U^{n}_{i,N_y,k} + b_y U^{n}_{i,N_y-1,k} +
  2 a_y h_y \left( q_{ik}^{n+1} + q_{ik}^{n} \right). \\
  \beta_3 U^{n+1}_{i,j,1} + \alpha_3 U^{n+1}_{i,j,0} &=
  b_z U^{n}_{i,j,1} + \left(1 - b_z - 2 c_z h_z p^{n}_{ij} \right) U^{n}_{i,j,0} -
  2 c_z h_z (q_{ij}^{n+1} + q_{ij}^{n}). \\
  \delta_3 U^{n+1}_{i,j,N_z} + \gamma_3 U^{n+1}_{i,j,N_z-1} &=
  \left(1 - b_z - 2 h_z a_z p^{n}_{ij}\right) U^{n}_{i,j,N_z} + b_z U^{n}_{i,j,N_z-1} +
  2 a_z h_z \left( q_{ij}^{n+1} + q_{ij}^{n} \right).
\end{align*}
где $a_x = \frac{\tau D^+_x}{2h^2}, c_x = \frac{\tau D^-_x}{2h^2}, b_x = a_x + c_x$, $a_y = \frac{\tau D^+_y}{2h^2}, c_y = \frac{\tau D^-_y}{2h^2}, b_y = a_y + c_y$ и $a_z = \frac{\tau D^+_z}{2h^2}, c_z = \frac{\tau D^-_z}{2h^2}, b_z = a_z + c_z$,
$\alpha_1 = 1 + b_x  + 2 c_x h_x p^{n+1}_{jk} $, $ \beta_1 = - b_x$, $\gamma_1 = -b_x$, $\delta_1 = 1 + b_x + 2 h_x a_x p^{n+1}_{jk}$, $\alpha_2 = 1 + b_y  + 2 c_y h_y p^{n+1}_{ik}$, $ \beta_2 = - b_y$, $\gamma_2 = -b_y$, $\delta_2 = 1 + b_y + 2 h_y a_y p^{n+1}_{ik}$, $\alpha_3 = 1 + b_z  + 2 c_z h_z p^{n+1}_{ij}$, $ \beta_3 = - b_z$, $\gamma_3 = - b_z$, $\delta_3 =  1 + b_z + 2 h_z a_z p^{n+1}_{ij}$.

\hrulefill

\begin{thebibliography}{99}
  \addcontentsline{toc}{section}{Список литературы}
\bibitem{Popov} Попов А.М.
  Вычислительные нанотехнологии. М.: КноРус, 2014. 312 с.
\end{thebibliography}
\end{document}
