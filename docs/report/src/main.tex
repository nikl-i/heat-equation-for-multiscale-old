\documentclass[a4paper,12pt]{article}
\usepackage[T2A]{fontenc}
\usepackage[utf8]{inputenc}
\usepackage[russianb,english]{babel}
\usepackage[pdftex,unicode]{hyperref}
\usepackage{amssymb,amsfonts,amsmath,mathtext,cite,enumerate,float,indentfirst}
\usepackage{geometry} % Меняем поля страницы
\usepackage{cmap} % Русский поиск в pdf
\usepackage{ccaption}
    \captiondelim{. } % после точки стоит пробел!
\geometry{left=2cm} % левое поле
\geometry{right=2cm}% правое поле
\geometry{top=2cm}% верхнее поле
\geometry{bottom=2cm}% нижнее поле
\usepackage{graphicx}
\graphicspath{{images/}}

\sloppy

\begin{document}
\renewcommand{\contentsname}{Содержание}
\renewcommand{\figurename}{Рис.}
\renewcommand{\bibname}{Список литературы}
\renewcommand{\refname}{Список литературы}
\renewcommand{\tablename}{Таблица}


\section{Схема Кранка -- Николсон}
\begin{equation}
  \label{eq:Crank-Nicolson}
    \frac{U^{n+1} - U^n}{\tau} = \frac{1}{2} L_1 U^{n+1}  + \frac{1}{2} L_1 U^n 
\end{equation}

Проведём разделение известных и неизвестных.
\begin{equation}
    \left( I - \frac{\tau}{2} L_1 \right) U^{n+1} = \left( I + \frac{\tau}{2} L_1 \right) U^n 
\end{equation}

Оператор имеет вид

\begin{equation*}
  L_1 U = \frac{D^+_x \frac{U_{i+1} - U_{i}}{h} - D^-_x\frac{U_{i} - U_{i-1}}{h}}{h}  - qU_{i} = \frac{D^+_x}{h^2}U_{i+1} - \left( \frac{D^+_x}{h^2} + \frac{D^-_x}{h^2} + q \right)U_{i} +  \frac{D^-_x}{h^2}U_{i-1}
\end{equation*}
Левая часть имеет вид
\begin{multline*}
    \left( I - \frac{\tau}{2} L_1 \right) U^{n+1} = U^{n+1}_i - \frac{\tau}{2} \left[ \frac{D^+_x}{h^2}U_{i+1} - \left( \frac{D^+_x}{h^2} + \frac{D^-_x}{h^2} + q \right) U^{n+1}_{i} +  \frac{D^-_x}{h^2}U^{n+1}_{i-1} \right] = \\
    - \frac{\tau D^+_x}{2h^2}U^{n+1}_{i+1} + \left(1 + \frac{\tau D^+_x}{2 h^2} + \frac{\tau D^-_x}{2 h^2} + \frac{\tau q}{2} \right) U^{n+1}_{i} - \frac{\tau D^-_x}{2h^2}U^{n+1}_{i-1}
\end{multline*}
Правая часть имеет вид
\begin{multline*}
    \left( I + \frac{\tau}{2} L_1 \right) U^{n} = U^{n}_i + \frac{\tau}{2} \left[ \frac{D^+_x}{h^2}U^{n}_{i+1} - \left( \frac{D^+_x}{h^2} + \frac{D^-_x}{h^2} + q \right) U^{n}_{i} +  \frac{D^-_x}{h^2}U^{n}_{i-1} \right] = \\
    \frac{\tau D^+_x}{2h^2}U^{n}_{i+1} + \left(1 - \left[\frac{\tau D^+_x}{2 h^2} + \frac{\tau D^-_x}{2 h^2} + \frac{\tau q}{2} \right] \right) U^{n}_{i} + \frac{\tau D^-_x}{2h^2}U^{n}_{i-1}
\end{multline*}
Окончательно имеем

\begin{multline*}
  - \frac{\tau D^+_x}{2h^2}U^{n+1}_{i+1} + \left(1 + \frac{\tau D^+_x}{2 h^2} + \frac{\tau D^-_x}{2 h^2} + \frac{\tau q}{2} \right) U^{n+1}_{i} - \frac{\tau D^-_x}{2h^2}U^{n+1}_{i-1}= \\
  = \frac{\tau D^+_x}{2h^2}U^{n}_{i+1} + \left(1 - \left[ \frac{\tau D^+_x}{2 h^2} + \frac{\tau D^-_x}{2 h^2} + \frac{\tau q}{2} \right] \right) U^{n}_{i} + \frac{\tau D^-_x}{2h^2}U^{n}_{i-1}
\end{multline*}


\section{Схема Дугласа -- Ганна для двумерной области}
\begin{equation}
  \begin{cases}
    \frac{U^* - U^n}{\tau} = \frac{1}{2} L_1 U^* + \frac{1}{2} L_1 U^n + L_2 U^n, \\
    \frac{U^{n+1} - U^n}{\tau} = \frac{1}{2} L_1 U^* + \frac{1}{2} L_1 U^n + \frac{1}{2} L_2 U^{n+1} + \frac{1}{2} L_2 U^n
  \end{cases}
\end{equation}

Проведём разделение известных и неизвестных переменных для каждого из уравнений.

\begin{equation}
  \begin{cases}
    \frac{U^*}{\tau} - \frac{1}{2} L_1 U^* = \frac{U^n}{\tau} + \frac{1}{2} L_1 U^n + L_2 U^n, \\
    \frac{U^{n+1}}{\tau} - \frac{1}{2} L_2 U^{n+1} = \frac{U^n}{\tau} + \frac{1}{2} L_1 U^* + \frac{1}{2} L_1 U^n + \frac{1}{2} L_2 U^n
  \end{cases}
\end{equation}


Сгруппируем операторы

\begin{equation}
  \begin{cases}
    \left( \frac{1}{\tau} I - \frac{1}{2} L_1 \right) U^* = \left( \frac{1}{\tau} I + \frac{1}{2} L_1 + L_2 \right) U^n, \\
    \left( \frac{1}{\tau} I - \frac{1}{2} L_2 \right) U^{n+1} = \left( \frac{1}{\tau} I  + \frac{1}{2} L_1 + \frac{1}{2} L_2 \right) U^n + \frac{1}{2} L_1 U^*
  \end{cases}
\end{equation}

Вычтем первое уравнение из второго:

\begin{equation}
  \begin{cases}
    \left( \frac{1}{\tau} I - \frac{1}{2} L_1 \right) U^* = \left( \frac{I}{\tau} + \frac{1}{2} L_1 + L_2 \right) U^n, \\
    \left( \frac{1}{\tau} I - \frac{1}{2} L_2 \right) U^{n+1} - \left( \frac{1}{\tau} I - \frac{1}{2} L_1 \right) U^* = - \frac{1}{2} L_2 U^n + \frac{1}{2} L_1 U^*
  \end{cases}
\end{equation}

Отсюда

\begin{equation}
  \begin{cases}
    \left( \frac{1}{\tau} I - \frac{1}{2} L_1 \right) U^* = \left( \frac{I}{\tau} + \frac{1}{2} L_1 + L_2 \right) U^n, \\
    \left( \frac{1}{\tau} I - \frac{1}{2} L_2 \right) U^{n+1} = \frac{1}{\tau} U^* - \frac{1}{2} L_2 U^n
  \end{cases}
\end{equation}
Домножим на $\tau$

\begin{equation}
  \label{eq:Douglas-Gunn-2d}
  \begin{cases}
    \left( I - \frac{\tau}{2} L_1 \right) U^* = \left( I + \frac{\tau}{2} L_1 + \tau L_2 \right) U^n, \\
    \left( I - \frac{\tau}{2} L_2 \right) U^{n+1} = U^* - \frac{\tau}{2} L_2 U^n, \\
  \end{cases}
\end{equation}

Операторы имеют вид:
\begin{equation}
  \begin{aligned}
    L_1 U = \frac{D^+_x \frac{U_{i+1,j} - U_{i,j}}{h} - D^-_x\frac{U_{i,j} - U_{i-1,j}}{h}}{h}  - qU_{i,j} = \\
    = \frac{D^+_x U_{i+1,j} + D^-_x U_{i-1,j}}{h^2} - \frac{D^+_x}{h^2}U_{i,j} - \frac{D^-_x}{h^2} U_{i,j} - q U_{i,j} = \\
    = \frac{D^+_x U_{i+1,j} + D^-_x U_{i-1,j}}{h^2} - \left(\frac{D^+_x + D^-_x}{h^2} + q \right) U_{i,j}, \quad \forall i,j; \\
    L_2 U = \frac{D^+_y U_{i,j+1} + D^-_y U_{i,j-1}}{h^2} - \left(\frac{D^+_y + D^-_y}{h^2} + q \right) U_{i,j}
  \end{aligned}
\end{equation}

Преобразуем левые части уравнений
\begin{multline*}
      \left( I - \frac{\tau}{2} L_1 \right) U^* = U^*_{i,j} - \frac{\tau}{2} \left[ \frac{D^+_x U^*_{i+1,j} + D^-_x U^*_{i-1,j}}{h^2} - \left(\frac{D^+_x + D^-_x}{h^2} + q \right) U^*_{i,j} \right] = \\
    - \frac{\tau D^+_x }{2 h^2} U^*_{i+1,j} + \left(1 + \frac{\tau D^+_x}{2h^2} + \frac{\tau D^-_x}{2 h^2} + \frac{\tau q}{2} \right) U^*_{i,j} - \frac{\tau D^-_x}{h^2} U^*_{i-1,j}
\end{multline*}

\begin{multline*}
        \left( I - \frac{\tau}{2} L_2 \right) U^{n+1} = U^{n+1}_{i,j} - \frac{\tau}{2} \left[ \frac{D^+_y U^{n+1}_{i,j+1} + D^-_y U^{n+1}_{i,j-1}}{h^2} - \left(\frac{D^+_y + D^-_y}{h^2} + q \right) U^{n+1}_{i,j} \right] = \\
    - \frac{\tau D^+_y }{2 h^2} U^{n+1}_{i,j+1} + \left(1 + \frac{\tau D^+_y}{2h^2} + \frac{\tau D^-_y}{2 h^2} + \frac{\tau q}{2} \right) U^{n+1}_{i,j} - \frac{\tau D^-_y}{h^2} U^{n+1}_{i,j-1}
\end{multline*}
Преобразуем правые части:

\begin{multline*}
  \left( I + \frac{\tau}{2} L_1 + \tau L_2 \right) U^n = U^n_{i,j} + \frac{\tau}{2} \left[ \frac{D^+_x U^n_{i+1,j} + D^-_x U^n_{i-1,j}}{h^2} - \left(\frac{D^+_x + D^-_x}{h^2} + q \right) U^n_{i,j} \right] + \\  + \tau \left[ \frac{D^+_y U^n_{i,j+1} + D^-_y U^n_{i,j-1}}{h^2} - \left(\frac{D^+_y + D^-_y}{h^2} + q \right) U^n_{i,j} \right] 
= \frac{\tau D^+_x}{2 h^2} U^n_{i+1,j} + \frac{\tau D^-_x}{2 h^2} U^n_{i-1,j} + \\ + \frac{\tau D^+_y}{h^2} U^n_{i,j+1} + \frac{\tau D^-_y}{h^2} U^n_{i,j-1} + \left[ 1 - \frac{\tau D^+_x}{2h^2} - \frac{\tau D^-_x}{2h^2} - \frac{ \tau D^+_y}{h^2} - \frac{\tau D^-_y}{h^2} - \frac{3 \tau q}{2} \right] U^n_{i,j}
\end{multline*}


\begin{multline*}
  U^* - \frac{\tau}{2} L_2 U^n = U_{i,j}^* - \frac{\tau}{2} \left[ \frac{D^+_y U_{i,j+1} + D^-_y U_{i,j-1}}{h^2} - \left(\frac{D^+_y + D^-_y}{h^2} + q \right) U_{i,j} \right] = \\ 
- \frac{\tau D^+_y}{2 h^2} U_{i,j+1} - \frac{\tau D^-_y}{2 h^2} U_{i,j-1} + \left( 1 + \frac{\tau D^+_y}{2 h^2} + \frac{\tau D^-_y}{2 h^2} + \frac{\tau q}{2} \right) U_{i,j}
\end{multline*}

В итоге




\section{Схема Дугласа -- Ганна}


Многошаговая реализация метода Дугласа -- Ганна имеет вид:
\begin{equation}
  \label{eq:Douglas-Gunn}
  \begin{cases}
    \frac{U^{*}   - U^n}{\tau} &= \frac{1}{2} L_1 U^{*}  + \frac{1}{2} L_1 U^n +  L_2 U^n +  L_3 U^n \\
    \frac{U^{**}  - U^n}{\tau} &= \frac{1}{2} L_1 U^{*}  + \frac{1}{2} L_1 U^n + 
                                \frac{1}{2} L_2 U^{**} + \frac{1}{2} L_2 U^n + L_3 U^n \\
    \frac{U^{n+1} - U^n}{\tau} &= \frac{1}{2} L_1 U^{*}   + \frac{1}{2} L_1 U^n +   
                                \frac{1}{2} L_2 U^{**}  + \frac{1}{2} L_2 U^n +  
                                \frac{1}{2} L_3 U^{n+1} + \frac{1}{2} L_3 U^n  

  \end{cases}
\end{equation}


Каждое из уравнений системы (\ref{eq:Douglas-Gunn}) --- аппроксимация полного уравнения диффузии => не нужно модификаций для граничных условий.
Проведём разделение известных и неизвестных переменных для каждого из уравнений.

\begin{equation*}
  \begin{cases}
    \frac{U^{*}}{\tau}  - \frac{1}{2} L_1 U^{*} &= \frac{U^n}{\tau} + \frac{1}{2} L_1 U^n +  L_2 U^n +  L_3 U^n \\
    \frac{U^{**}}{\tau} - \frac{1}{2} L_2 U^{**} &= \frac{U^n}{\tau} + \frac{1}{2} L_1 U^{*}  + \frac{1}{2} L_1 U^n + \frac{1}{2} L_2 U^n + L_3 U^n \\
    \frac{U^{n+1}}{\tau} - \frac{1}{2} L_3 U^{n+1} &= \frac{U^n}{\tau} + \frac{1}{2} L_1 U^{*}   + \frac{1}{2} L_1 U^n + \frac{1}{2} L_2 U^{**}  + \frac{1}{2} L_2 U^n + \frac{1}{2} L_3 U^n  
  \end{cases}
\end{equation*}

Сгруппируем операторы
\begin{equation}
  \label{eq:Douglas-Gunn3}
  \begin{cases}
    \left( \frac{1}{\tau} I - \frac{1}{2} L_1 \right) U^{*} &= 
        \left( \frac{1}{\tau} I + \frac{1}{2} L_1 +  L_2 + L_3 \right) U^n \\
    \left( \frac{1}{\tau} I - \frac{1}{2} L_2 \right) U^{**} &= 
        \left( \frac{1}{\tau} I + \frac{1}{2} L_1 + \frac{1}{2} L_2 + L_3 \right) U^n + \frac{1}{2} L_1 U^{*}\\
    \left( \frac{1}{\tau} I - \frac{1}{2} L_3 \right) U^{n+1} &= 
        \left( \frac{1}{\tau} I  + \frac{1}{2} L_1 + \frac{1}{2} L_2 + \frac{1}{2} L_3 \right) U^n + \frac{1}{2} L_1 U^{*}   + \frac{1}{2} L_2 U^{**}  
  \end{cases}
\end{equation}

В системе (\ref{eq:Douglas-Gunn3}) вычтем первое уравнение из второго и второе из третьего:

\begin{equation*}
  \begin{cases}
      &\left( \frac{1}{\tau} I - \frac{1}{2} L_1 \right) U^{*} =
      \left( \frac{1}{\tau} I + \frac{1}{2} L_1 +  L_2 + L_3 \right) U^n \\
      &\left( \frac{1}{\tau} I - \frac{1}{2} L_2 \right) U^{**} - 
      \left( \frac{1}{\tau} I - \frac{1}{2} L_1 \right) U^{*} = \\
      & \qquad = \left( \frac{1}{\tau} I + \frac{1}{2} L_1 + \frac{1}{2} L_2 + L_3 \right) U^n + \frac{1}{2} L_1 U^{*} - \\
      & \qquad - \left( \frac{1}{\tau} I + \frac{1}{2} L_1 +  L_2 + L_3 \right) U^n \\
      &\left( \frac{1}{\tau} I - \frac{1}{2} L_3 \right) U^{n+1} -
      \left( \frac{1}{\tau} I - \frac{1}{2} L_2 \right) U^{**} = \\
      & \qquad = \left( \frac{1}{\tau} I + \frac{1}{2} L_1 + \frac{1}{2} L_2 + \frac{1}{2} L_3 \right) U^n + \frac{1}{2} L_1 U^{*}   + \frac{1}{2} L_2 U^{**} - \\
      & \qquad - \left( \frac{1}{\tau} I + \frac{1}{2} L_1 + \frac{1}{2} L_2 + L_3 \right) U^n - \frac{1}{2} L_1 U^{*}
  \end{cases}
\end{equation*}

Отсюда
\begin{equation*}
  \begin{cases}
      \left( \frac{1}{\tau} I - \frac{1}{2} L_1 \right) U^{*} &=
      \left( \frac{1}{\tau} I + \frac{1}{2} L_1 +  L_2 + L_3 \right) U^n \\
      \left( \frac{1}{\tau} I - \frac{1}{2} L_2 \right) U^{**} -
      \left( \frac{1}{\tau} I - \frac{1}{2} L_1 \right) U^{*} &= - \frac{1}{2} L_2 U^n + \frac{1}{2} L_1 U^{*} \\
      \left( \frac{1}{\tau} I - \frac{1}{2} L_3 \right) U^{n+1} - 
      \left( \frac{1}{\tau} I - \frac{1}{2} L_2 \right) U^{**} &= - \frac{1}{2} L_3 U^n + \frac{1}{2} L_2 U^{**}
  \end{cases}
\end{equation*}

Переносим вычитаемые в правую часть и сокращаем
\begin{equation*}
  \begin{cases}
      \left( \frac{1}{\tau} I - \frac{1}{2} L_1 \right) U^{*} &=
      \left( \frac{1}{\tau} I + \frac{1}{2} L_1 +  L_2 + L_3 \right) U^n \\
      \left( \frac{1}{\tau} I - \frac{1}{2} L_2 \right) U^{**} & =
      \frac{1}{\tau} U^{*} - \frac{1}{2} L_2 U^n \\
      \left( \frac{1}{\tau} I - \frac{1}{2} L_3 \right) U^{n+1} & = 
      \frac{1}{\tau} U^{**} - \frac{1}{2} L_3 U^n
  \end{cases}
\end{equation*}

Домножим все части на $\tau$:
\begin{equation*}
  \label{eq:system}
  \begin{cases}
      \left( I - \frac{\tau}{2} L_1 \right) U^{*} &=  
          \left( I + \frac{\tau}{2} L_1 +  \tau L_2 + \tau L_3 \right) U^n \\
      \left( I - \frac{\tau}{2} L_2 \right) U^{**} & = U^{*} - \frac{\tau}{2} L_2 U^n \\
      \left( I - \frac{\tau}{2} L_3 \right) U^{n+1} & =  U^{**} - \frac{\tau}{2} L_3 U^n
  \end{cases}
\end{equation*}

Операторы $L_1$, $L_2$ и $L_3$ имеют вид:
\begin{equation}
  \begin{aligned}
    L_1 U = \frac{D^+_x \frac{U_{i+1,j,k} - U_{i,j,k}}{h} - D^-_x\frac{U_{i,j,k} - U_{i-1,j,k}}{h}}{h}  - qU_{i,j,k} = \\
    = \frac{D^+_x U_{i+1,j,k} + D^-_x U_{i-1,j,k}}{h^2} - \frac{D^+_x}{h^2}U_{i,j,k} - \frac{D^-_x}{h^2} U_{i,j,k} - q U_{i,j,k} = \\
    = \frac{D^+_x U_{i+1,j,k} + D^-_x U_{i-1,j,k}}{h^2} - \left(\frac{D^+_x + D^-_x}{h^2} + q \right) U_{i,j,k}, \quad \forall i,j,k; \\
    L_2 U = \frac{D^+_y U_{i,j+1,k} + D^-_y U_{i,j-1,k}}{h^2} - \left(\frac{D^+_y + D^-_y}{h^2} + q \right) U_{i,j,k} \\
    L_3 U = \frac{D^+_z U_{i,j,k+1} + D^-_z U_{i,j,k-1}}{h^2} - \left(\frac{D^+_z + D^-_z}{h^2} + q \right) U_{i,j,k}
  \end{aligned}
\end{equation}
Перепишем левые части уравнений системы (\ref{eq:system}) в виде, удобном для заполнения матриц

\begin{equation}
  \begin{aligned}
   \left( I - \frac{\tau}{2} L_1 \right) U^{*} = U^*_{i,j,k} - \frac{\tau}{2} \left[ \frac{D^+_x U^*_{i+1,j,k} + D^-_x U^*_{i-1,j,k}}{h^2} - \left(\frac{D^+_x + D^-_x}{h^2} + q \right) U^*_{i,j,k} \right] = \\
= - \frac{\tau D^+_x}{2h^2} U^{*}_{i+1,j,k} + \left( 1 + \frac{\tau}{2} \left[ \frac{D^+_x + D^-_x}{h^2} + q \right] \right) U^{*}_{i,j,k} - \frac{\tau D^-_x}{2 h^2} U^{*}_{i-1,j,k} = \\
= - \frac{\tau D^+_x}{2h^2} U^{*}_{i+1,j,k} + \left( 1 + \frac{\tau D^+_x}{2h^2} + \frac{\tau D^-_x}{2h^2} + \frac{\tau q}{2} \right) U^{*}_{i,j,k} - \frac{\tau D^-_x}{2 h^2} U^{*}_{i-1,j,k} \\
   \left( I - \frac{\tau}{2} L_2 \right) U^{**} = - \frac{\tau D^+_y}{2h^2} U^{**}_{i,j+1,k} + \left( 1  + \frac{\tau D^+_y}{2h^2} + \frac{\tau D^-_y}{2h^2} + \frac{\tau q}{2} \right) U^{**}_{i,j,k} - \frac{\tau D^-_y}{2 h^2} U^{**}_{i,j-1,k}\\
   \left( I - \frac{\tau}{2} L_3 \right) U^{n+1} = - \frac{\tau D^+_z}{2h^2} U^{n+1}_{i,j,k+1} + \left( 1 + \frac{\tau D^+_z}{2h^2} + \frac{\tau D^-_z}{2h^2} + \frac{\tau q}{2} \right) U^{n+1}_{i,j,k} - \frac{\tau D^-_z}{2 h^2} U^{n+1}_{i,j,k-1}
  \end{aligned}
\end{equation}

Распишем правую часть первого уравнения системы (\ref{eq:system})
% \begin{equation}
\begin{multline}
  \left( I + \frac{\tau}{2} L_1 +  \tau L_2 + \tau L_3 \right) U 
  = U_{i,j,k} + \frac{\tau}{2} \left[ \frac{D^+_x U_{i+1,j,k} + D^-_x U_{i-1,j,k}}{h^2} - \left(\frac{D^+_x + D^-_x}{h^2} + q \right) U_{i,j,k} \right] +  \\
  + \tau \left[ \frac{D^+_y U_{i,j+1,k} + D^-_y U_{i,j-1,k}}{h^2} - \left(\frac{D^+_y + D^-_y}{h^2} + q \right) U_{i,j,k} \right] + \\
  + \tau \left[ \frac{D^+_z U_{i,j,k+1} + D^-_z U_{i,j,k-1}}{h^2} - \left(\frac{D^+_z + D^-_z}{h^2} + q \right) U_{i,j,k} \right] = \\
  = \frac{\tau}{2} \left[ \frac{D^+_x U_{i+1,j,k} + D^-_x U_{i-1,j,k}}{h^2} \right] + \tau \left[\frac{D^+_y U_{i,j+1,k} + D^-_y U_{i,j-1,k}}{h^2} \right] + \tau \left[ \frac{D^+_z U_{i,j,k+1} + D^-_z U_{i,j,k-1}}{h^2} \right] + \\
  + \left[ 1 - \frac{\tau}{2} \left(\frac{D^+_x + D^-_x}{h^2} + q \right) - \tau \left(\frac{D^+_y + D^-_y}{h^2} + q \right) - \tau \left(\frac{D^+_z + D^-_z}{h^2} + q \right) \right] U_{i,j,k} = \\
  = \frac{\tau D^+_x}{2h^2} U_{i+1,j,k} + \frac{\tau D^-_x}{2h^2} U_{i-1,j,k} + \frac{\tau D^+_y}{h^2} U_{i,j+1,k} + \frac{\tau D^-_y}{h^2} U_{i,j-1,k} + \frac{\tau D^+_z}{h^2} U_{i,j,k+1} + \frac{\tau D^-_z}{h^2} U_{i,j,k-1} + \\
  + \left[ 1 - \left( \frac{\tau D^+_x}{2h^2} + \frac{\tau D^-_x}{2h^2} + \frac{\tau D^+_y}{h^2} + \frac{\tau D^-_y}{h^2} + \frac{\tau D^+_z}{h^2} + \frac{\tau D^-_z}{h^2} + \frac{5 \tau q}{2} \right) \right] U_{i,j,k} , \quad \forall i,j,k; \\
\end{multline}
% \end{equation}
То же для остальных уравнений:
\begin{equation*}
  \begin{aligned}
    U^{*} - \frac{\tau}{2} L_2 U = U^{*}_{i,j,k} - \frac{\tau}{2} \left[ \frac{D^+_y U_{i,j+1,k} + D^-_y U_{i,j-1,k}}{h^2} - \left(\frac{D^+_y + D^-_y}{h^2} + q \right) U_{i,j,k} \right] = \\
 = U^{*}_{i,j,k} - \frac{\tau D^+_y}{2 h^2} U_{i,j+1,k} - \frac{\tau D^-_y}{2 h^2} U_{i,j-1,k} + \left( \frac{\tau D^+_y}{2 h^2} + \frac{\tau D^-_y}{2 h^2} + \frac{\tau q}{2}  \right) U_{i,j,k} \\
    U^{**} - \frac{\tau}{2} L_3 U = U^{**}_{i,j,k} - \frac{\tau}{2} \left[ \frac{D^+_z U_{i,j,k+1} + D^-_z U_{i,j,k-1}}{h^2} - \left(\frac{D^+_z + D^-_z}{h^2} + q \right) U_{i,j,k} \right] = \\
    = U^{**}_{i,j,k} - \frac{\tau D^+_z}{2h^2} U_{i,j,k+1} - \frac{\tau D^-_z}{2h^2} U_{i,j,k-1} + \left(\frac{\tau D^+_z}{2h^2} +\frac{\tau D^-_z}{2h^2} + \frac{\tau q}{2} \right) U_{i,j,k}
  \end{aligned}
\end{equation*}

Окончательно на $n$-ом шаге имеем:

\begin{multline}
  - \frac{\tau D^+_x}{2h^2} U^{*}_{i+1,j,k} + \left( 1 + \frac{\tau D^+_x}{2h^2} + \frac{\tau D^-_x}{2h^2} + \frac{\tau q}{2} \right) U^{*}_{i,j,k} - \frac{\tau D^-_x}{2 h^2} U^{*}_{i-1,j,k} = \\
  = \frac{\tau D^+_x}{2h^2} U^n_{i+1,j,k} + \frac{\tau D^-_x}{2h^2} U^n_{i-1,j,k} + \frac{\tau D^+_y}{h^2} U^n_{i,j+1,k} + \frac{\tau D^-_y}{h^2} U^n_{i,j-1,k} + \frac{\tau D^+_z}{h^2} U^n_{i,j,k+1} + \frac{\tau D^-_z}{h^2} U^n_{i,j,k-1} + \\
  + \left[ 1 - \left( \frac{\tau D^+_x}{2h^2} + \frac{\tau D^-_x}{2h^2} + \frac{\tau D^+_y}{h^2} + \frac{\tau D^-_y}{h^2} + \frac{\tau D^+_z}{h^2} + \frac{\tau D^-_z}{h^2} + \frac{5 \tau q}{2} \right) \right] U^n_{i,j,k} \\
\end{multline}
\begin{multline}
  - \frac{\tau D^+_y}{2h^2} U^{**}_{i,j+1,k} + \left( 1 + \frac{\tau D^+_y}{2h^2} + \frac{\tau D^-_y}{2h^2} + \frac{\tau q}{2} \right) U^{**}_{i,j,k} - \frac{\tau D^-_y}{2 h^2} U^{**}_{i,j-1,k} = \\
  = U^{*}_{i,j,k} - \frac{\tau D^+_y}{2 h^2} U^n_{i,j+1,k} - \frac{\tau D^-_y}{2 h^2} U^n_{i,j-1,k} + \left( \frac{\tau D^+_y}{2 h^2} + \frac{\tau D^-_y}{2 h^2} + \frac{\tau q}{2}  \right) U^n_{i,j,k}
\end{multline}
\begin{multline}
  - \frac{\tau D^+_z}{2h^2} U^{n+1}_{i,j,k+1} + \left( 1 + \frac{\tau D^+_z}{2h^2} + \frac{\tau D^-_z}{2h^2} + \frac{\tau q}{2} \right) U^{n+1}_{i,j,k} - \frac{\tau D^-_z}{2 h^2} U^{n+1}_{i,j,k-1} = \\
  = U^{**}_{i,j,k} - \frac{\tau D^+_z}{2h^2} U^n_{i,j,k+1} - \frac{\tau D^-_z}{2h^2} U^n_{i,j,k-1} + \left(\frac{\tau D^+_z}{2h^2} +\frac{\tau D^-_z}{2h^2} + \frac{\tau q}{2} \right) U^n_{i,j,k}
\end{multline}
% \begin{itemize}
% \item Запрогать
% \item Граничные условия 
% \item Строить профиль as matrix 
% \item Сравнение с аналитическим решением
% \item Сравнить циклическую редукцию с остальными методами. Какая сложность?
% \item Дописать выкладки
% \item Переменный D и ключи --- тип задачи
% \item Реализовать всё на фортране
% \item Радиальная гистограмма?
% \item Монитор: на можем читать из файла одновременно со счётом, нет гарантий получения узлов на счёт вовремя
% \item Метод зонтика для FES Метод взвешенных гистограмм
% \end{itemize}



% Входные данные:
% размеры шагов по времени $\tau$ и пространству $h$, коэффициенты $D$, $q = \sigma(s) \cdot E^2$, начальное условия $U(r, 0) = U_0(r)$, граничные условия $U|_\Omega = g(r,t)$



% \begin{thebibliography}{99}
% \addcontentsline{toc}{section}{Список литературы}
% \bibitem{Popov} Попов А.М. 
%   Вычислительные нанотехнологии. М.: КноРус, 2014. 312 с.
% \end{thebibliography}
\end{document}


